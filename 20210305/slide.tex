\documentclass[10pt]{beamer}
\usepackage{xeCJK}
\usepackage{graphicx}
\usepackage{booktabs}
\usepackage{listings}
\usepackage{multirow}
\usepackage{mathtools}
\usepackage{ulem}
\usefonttheme[onlymath]{serif}
\usetheme{metropolis}
\begin{document}

\title{Water Problem Choose Talk}
\date{\today}
\maketitle
\clearpage

\begin{frame}
	\frametitle{WC2021 Day-1 T1}

	一个由$nmk$个小正方体组成的体积为奇数的立方体,你在$(1,1,1)$,请找到一条到$(\frac{n+1}2,\frac{m+1}2,\frac{k+1}2)$的路径(每个小正方体可以走到和它有公共面的小正方体),经过所有立方体恰好一次。

	无解输出$-1$。

	$nmk\le 10^5$。

	Subtask: $n=1$/$n=m=k$。
\end{frame}
\clearpage
\begin{frame}
	\frametitle{WC2021 Day-1 T2}

	给定$n,a_{1\dots n-1},p_{1\dots n}$,求$b_{1\dots n}$,满足$b_ib_{i+1}\ge a_i$,求下式最小值:

	$$
	\sum_{i=1}^n b_ip_i
	$$

	$n\le 2000$。

	Subtask:$p_i=1$/$a_i=1$/$n\le 20$/$n\le 200$。
\end{frame}
\clearpage
\begin{frame}
	\frametitle{WC2021 Day0 T1}

	给定一个无向简单联通图,满足图内对于所有点数多于$3$的简单环,存在一条不在环上的连接环上两点的边。

	保证它的最大团大小为偶数。
	
	将点分为两个点集$A,B$,使$A,B$导出子图最大团大小相等。

	节点数$\le 10^5$

	无解输出$-1$。
\end{frame}
\clearpage

\begin{frame}
	\frametitle{WC2021 Day0 T3}

	给定一棵树,第$i$条边连接$u_i,v_i$,且有两个权$a_i,b_i$。

	一个人从$u_i$走到$v_i$(或$v_i$走到$u_i$)需要$a_i$天,且每天$u_i$,$v_i$最多$b_i$人走到第$i$条边上(可以$u_i$进入$b_i$人,同时$v_i$进入$b_i$人)。

	给定每个点的人口,求最快需要几天,他们可以到达同一个点,或者走到同一条边上。

	节点数$\le 10^5$。

	Subtask: $b_i=1$。

\end{frame}
\clearpage

\begin{frame}
	\frametitle{AGC002F}

	\url{https://atcoder.jp/contests/agc002/tasks/agc002_f}
\end{frame}
\clearpage

\begin{frame}
	\frametitle{AGC002F solution}

	显然最后有$k-1$个$1\sim n$和$n$个$0$。
	
	从后往前放,记$f_{i,j}$表示放了$i$个$0$,且第$i$个$0$放在位置$j$。

	转移枚举上一个$0$的位置,复杂度$O(n^3)$,前缀和优化成$O(n^2)$。

\end{frame}
\clearpage
\begin{frame}
	\frametitle{UOJ449 喂鸽子}

	\url{https://uoj.ac/problem/449}

\end{frame}
\clearpage
\begin{frame}
	\frametitle{solution}

	min-max 反演,变成某个集合最早喂饱的。

	只要求出$i$个鸽子的集合,喂$j$次依然没饱的概率即可。
	
	dp即可。
\end{frame}
\clearpage
\begin{frame}
	\frametitle{AGC020E}

	\url{https://atcoder.jp/contests/agc020/tasks/agc020_e}

\end{frame}
\clearpage
\begin{frame}
	\frametitle{AGC023E}

	\url{https://atcoder.jp/contests/agc023/tasks/agc023_e}

\end{frame}
\clearpage
\begin{frame}
	\frametitle{AGC024E}

	\url{https://atcoder.jp/contests/agc024/tasks/agc024_e}

\end{frame}
\clearpage
\begin{frame}
	\frametitle{AGC030D}

	\url{https://atcoder.jp/contests/agc030/tasks/agc030_d}

\end{frame}

\end{document}