\documentclass[10pt]{beamer}
\usepackage{xeCJK}
\usepackage{graphicx}
\usepackage{booktabs}
\usepackage{listings}
\usepackage{multirow}
\usepackage{mathtools}
\usepackage{ulem}
\usefonttheme[onlymath]{serif}
\usetheme{metropolis}
\begin{document}
	\title{杂题水讲}
	\date{\today}
	\author{huhao}
	\maketitle
	\clearpage
	\begin{frame}
		\frametitle{前言}
	
		\begin{itemize}
			\item juju都会,不会的可以问juju。
			\item 一共$ $道题,全是CF的,难度平均值为$ $。
		\end{itemize}
	
	\end{frame}
	\clearpage
	\begin{frame}
		\frametitle{CF1007 C. Guess two numbers}
		交互,两个数$a,b\in[1,10^{18}]$,你每次可以询问$x,y$,会根据条件回答(满足多个只会回答一个):
		\begin{enumerate}
			\item $x<a$
			\item $y<b$
			\item $x>a$或$y>b$
		\end{enumerate}
		你询问$x=a,y=b$时通过此题,次数限制$600$。
	\end{frame}
	\clearpage
	\begin{frame}
		\frametitle{solution (*3000)}
	
		显然,一开始满足所有询问的是一个正方形,然后若干次询问后变为一个长方形或一个L型,二分求解是$O(\log^2)$的。

		注意到有些询问二分的话对面积的消减比较小,可以在$0.8$处询问,这样可以让常数变得比较小,足以通过。
	
	\end{frame}
	\clearpage
	\begin{frame}
		\frametitle{CF1491 H. Yuezheng Ling and Dynamic Tree}
		给定$n-1$个数$f_{2\dots n}$,有两个操作(共$q$次):
		\begin{itemize}
			\item 给定$l,r,x$,令$f_{l\dots r}$减去$x$,并对$1$取较大值。
			\item 给定$u,v$,将$i,f_i$连边,形成$1$作为根的树,求$u,v$的lca。
		\end{itemize}
		$n,q\le 10^5$,1.5s。
	\end{frame}
	\clearpage
	\begin{frame}
		\frametitle{solution (*3400)}
	
		似乎没有好的$O(n~\mathrm{poly}(\log))$做法,考虑$O(n^{1.5})$。

		lca可以$O(\sqrt n)$求,不妨记两个数组:$f,F$,$f$是父亲,$F$是一个祖先。

		分为$\sqrt n$块,同时记$F$为不在同一块内的最深的祖先,不难发现求lca是$O(\sqrt n)$的。

		考虑对一整个块求$F$是$O(\sqrt n)$的,这个过程记为$1$次build。

		修改时,对两个散块进行build,复杂度是$O(q\sqrt n)$的。

		考虑到每一个块,如果对整块修改了$\sqrt n$次,那么$F=f$,就只要维护$f$了。

		均摊下来,一共进行$\sqrt n\times \sqrt n=n$次build,总复杂度$O((n+q)\sqrt n)$。
	
	\end{frame}
	\clearpage
	\begin{frame}
		\frametitle{CF1548 E. Gregor and the Two Painters}
	
		给定$n,m,a_{1\dots n},b_{1\dots m},x$,生成一个$n\times m$的黑白矩阵,$(i,j)$为黑当且仅当$a_i+b_j\le x$。
		
		求黑色连通块数。

		$n,m,a,b\in [1,200000]$。
	
	\end{frame}
	\clearpage
	\begin{frame}
		\frametitle{solution (*3400)}
	
		对于一个连通块只在$a_i+b_j$最小的格子处统计,相等统计$i,j$最大的。

		如果一个格子$(i,j)$所在连通块$(x,y)$被统计到,一定有$(i,y)$或$(x,j)$的值大于$(i,j)$,且与$(i,j)$联通。

		不难发现,可以得到$n+m$个区间$B_{1\dots n},A_{1\dots m}$,$(i,j)$被统计当且仅当$a_i\in A_j,b_j\in B_i$,扫描线+简单数据结构维护即可。
	
	\end{frame}
	\clearpage
	\begin{frame}
		\frametitle{CF891 E. Lust}
	
		$n$个数,$m$次操作,每次选$1$个数,记录下除它外所有数的乘积,并将它减1。

		对所有操作方案所有记录下的数求和,模$10^9+7$。

		$n\le 5000,k\le 10^9$
	
	\end{frame}
	\clearpage
	\begin{frame}
		\frametitle{solution (*3000)}
	
		不妨先考虑这些数等于0的情况,令$f_{m,n}$表示所有数成绩的和(除以$(-1)^mn!$)。

		$$
		f_{m,n}=\sum_i \dfrac{i}{i!}f_{m-1,n-i}
		$$

		不难发现:

		$$
		e^{x}x=\sum_i \dfrac{1}{i!}x^{i+1}=\sum_i \dfrac{i}{i!}x^i
		$$

		令$[x^n]F_m(x)=f_{m,n}$,则:
		
		$$
		F_m(x)=F_{m-1}(x)e^xx=x^me^{mx}
		$$
	
	\end{frame}
	\clearpage
	\begin{frame}
		\frametitle{solution}
	
		也就是说$f_{m,n}=m^{n-m}$。

		然后令最终数组$c_i=a_i-b_i$,只要考虑乘多少个$b_i$,可以通过dp来确定系数。

		然后枚举不计入乘积的数,通过上面dp的逆过程算出其它的数的dp值即可。

		另外需要注意一些细节,复杂度$O(n^2)$。

	\end{frame}
	\clearpage
	\begin{frame}
		\frametitle{CF1336 F. Journey}

		一棵树,若干条路径,求有多少对路径交点数$\ge k+1$。

		$n,q\le 1.5\times 10^5,k\le n$,4s。

		不会可以想一想下面两个简化版:

		\begin{enumerate}
			\item 树是条链。
			\item $k\le 10$。
		\end{enumerate}

	\end{frame}
	\clearpage
	\begin{frame}
		\frametitle{solution 1}
	
		考虑两条路径的交点,如果点数为$x$,那么有:

		长为$k$的公共路径有$x-k+1$条。

		长为$k+1$的公共路径有$x-k$条。

		但$x<k$时,上述两个都为$0$。

		不难发现,用上述两个值相减,就等于$[x\ge k]$。
	
	\end{frame}
	\clearpage
	\begin{frame}
		\frametitle{solution 1}
	
		问题转化为,求每两条路径的长为$k$的公共路径的数量和。
	
		不难发现,如果一条路径是$a_{1\dots m}$,那么将它拆成两条路径$a_{1\dots k},a_{2\dots m}$,答案不变。同理,拆成$m-k+1$条路径$a_{i\dots i+k-1}$答案也不变。

		对链上每个位置记录一下这有多少条路径即可。

	\end{frame}
	\clearpage
	\begin{frame}
		\frametitle{solution 2}
	
		同理,可以拆成若干条直上直下的路径和一条长为$2k-1$的路径。

		$k$很小,暴力拆开即可。
	
	\end{frame}
	\clearpage
	\begin{frame}
		\frametitle{solution (*3500)}
	
		现在的问题是lca处长为$2k-1$的路径不好处理。

		之前所有推论都没有指定根,不妨指定重心为根。

		先把所有路径尽可能拆开,剩下的分为两种:lca为重心的和不为重心的。

		lca不为重心的路径都在同一颗子树内,可以递归子树,这和点分治是一样的。

		只要处理lca为重心的即可。

		显然,可以在遍历树时用启发式合并。

		复杂度$O(n\log^2 n)$。
	
	\end{frame}
\end{document}