\documentclass[10pt]{beamer}
\usepackage{xeCJK}
\usepackage{graphicx}
\usepackage{booktabs}
\usepackage{listings}
\usepackage{multirow}
\usepackage{mathtools}
\usepackage{ulem}
\usefonttheme[onlymath]{serif}
\usetheme{metropolis}
\setbeamercolor{footnote mark}{fg=blue}% set footnote color in beamer
\setbeamerfont{footnote}{size=\tiny}
\begin{document}
	\title{杂题水讲}
	\date{\today}
	\author{huhao}
	\maketitle
	\begin{frame}
		\frametitle{开胃题}

		一张$n$个点的简单无向图,若没有三元环,边数最多有多少。
	\end{frame}
	\begin{frame}
		\frametitle{solution}
	
		显然是$\dfrac {n^2}4$,完全二分图就是一个合法的构造。

		但是怎么证明不存在边更多的方案?
	
	\end{frame}
	\begin{frame}
		\frametitle{proof}
	
		\newtheorem{lemma1}{引理:}[section]
		\begin{lemma1}
			$$
			\sum_{i} d_i^2\le nm
			$$

			证明:考虑每一条边两个端点连出的边数和,显然不大于$n$。
		\end{lemma1}

		所以:

		$$
		\dfrac{4m^2}{n}=\sum_i (\dfrac{2m}n)^2\le \sum_id_i^2\le nm
		$$

		所以$m\le \dfrac{n^2}{4}$。
	\end{frame}
	\begin{frame}
		\frametitle{博弈}
	
		大小为$a,b$的两堆石头,每次可以分别在两堆中取出$c,d$个,要求$|ad-bc|=1$,不能取的输,判断必胜状态。
	
	\end{frame}
	\begin{frame}
		\frametitle{solution}
	
		只考虑$\gcd(a,b)=1$:$a<b$时,$2|a$必败,否则必胜。
	
	\end{frame}
	\begin{frame}
		\frametitle{简单DP}
	
		
		小$l$发明了一种可以在日后与图灵机相提并论的机器,因为他特别强,所以他不需要你帮他设计机器,只需要统计出有多少个$n$个点的机器即可,每一个机器都如下:
		\begin{itemize}
		\item 机器由点、有向边以及无相边构成,有且仅有两个有向边不连接图中两点:一条只有出点,一条只有入点。
		\item 所有点有且仅有一条出边,一条入边,一条无相边与它相连。
		\item 所有边均不是自环,由上一条可以发现 不可能存在重边。
		\item 点与点,边与边均不可区分。
		\item 若将图的有向边视为无相边,则图联通。
		\end{itemize}

		$n\le 5000$。
	
	\end{frame}
	\begin{frame}
		\frametitle{solution}
	
		考虑给每一个点一个唯一的标号。
	
	\end{frame}
	\begin{frame}
		\frametitle{sum}
	
		求:

		$$
		(\sum_i \lfloor\dfrac ni\rfloor+\lfloor\sqrt n\rfloor)\bmod 2
		$$
	
	\end{frame}
	\begin{frame}
		\frametitle{博弈题}
	
		在$(2n)^2$的网格里有$1\sim (2n)^2$的数,AB两人依次取数,除了第一次外都必须与之前至少一次取数位置相邻。

		如果一个人取到的数总和较大,那么他就获胜了。AB两人足够聪明。

		试构造方案,使得A/B获胜。
	
	\end{frame}
	\begin{frame}
		\frametitle{NOI2016 旷野大计算}
	
		休息一下
		
		\url{https://www.luogu.com.cn/problem/P1737}
	
	\end{frame}
	\begin{frame}
		\frametitle{构造简单函数}

		$\rm sgn$:

		$$
		-1+2\lim_{t\rightarrow +\infty}s(2^tx)=\mathrm{sgn}(x)
		$$

		$I(x)=x$:

		不难发现$4s(x)=2+x+o(x),x\rightarrow 0$。

		所以
		$$
		c^{-1}(4s(cx)-2)=x+o(1),c\rightarrow 0
		$$

		$[op]x$:

		$$
		c^{-1}(4s(cx-\infty)-2)+c^{-1}(4s(cx+\infty)-2)=0
		$$

		$|x|$:

		组合一下$\mathrm{sgn},[op]x$即可。
		
	\end{frame}
	\begin{frame}
		\frametitle{solution}
	
		\sout{拆一下二进制就做完了吧。}

		那个乘以$0.1$就找一条斜率$0.1$的切线就好了。
	
	\end{frame}
	\begin{frame}
		\frametitle{AGC011E}
	
		定义一个数是递增的,当且仅当它每一位数字都比前面的大。

		给定一个大整数($10^{10^6}$级别),输出它最少是几个递增的数的和。
	
	\end{frame}
	\begin{frame}
		\frametitle{AGC011E}
	
		不难发现,一个递增的数可以表示为:

		$$
		\sum_{i=1}^9\dfrac{10^{p_i}-1}{9}
		$$

		简单讨论一下即可。
	
	\end{frame}
	\begin{frame}
		\frametitle{Rascal 三角}
	
		定义Rascal三角为:

		\begin{itemize}
			\item 第$i$行有$i$个元素,第一个和最后一个是$1$。
			\item 位置为$(i,j)$的左上方,右上方,正上方的元素分别为:$(i-1,j-1),(i-1,j),(i-2,j-1)$。
			\item 每一个元素都是它左上方乘以右上方加一,然后除以正上方的元素。
		\end{itemize}

		试找出Rascal三角的通项公式。
	
	\end{frame}
	\begin{frame}
		\frametitle{Det}
	
		定义$f_{1,i}=f_{i,1}=1,f_{i,j}=f_{i-1,j}+f_{i,j-1}$($i,j\le n$)。

		求$\det f$。
	
	\end{frame}
	\begin{frame}
		\frametitle{Swap}
	
		你有一个数组$f_{1\dots n},f_i=(i\bmod n)+1$,你想通过若干次swap使得$f_i=i$。

		但是对于两个指针只能swap一次,特别的,一开始$f_i,f_{(i\bmod n)+1}$间不能使用swap。

		你可以引入若干新的变量来进行辅助,试找到方案,使得引入变量个数不超过$2$。
	
	\end{frame}
	\begin{frame}
		\frametitle{solution}
	
2  3  4  5  6  …  k   1  x  y 

x  3  4  5  6  …  k   1  2  y

x  y  4  5  6  …  k   1  2  3

x  y  3  5  6  …  k   1  2  4

x  y  3  4  6  …  k   1  2  5

x  y  3  4  5  …  k   1  2  6

… … …

x  y  3  4  5  … k-1  1  2  k

x  y  3  4  5  … k-1  k  2  1

x  2  3  4  5  … k-1  k  y  1

1  2  3  4  5  … k-1  k  y  x
	
	\end{frame}
	\begin{frame}
		\frametitle{Guess 1}
	
		有$2^n$个人,背后都有$1\sim 2^n$的数,每个人都可以看到别人的数。

		构造方案,使得至少一个人可以正确的猜出他背后的数。
	
	\end{frame}
	\begin{frame}
		\frametitle{Guess 2}
	
		有$n$个人,背后都有一个实数,每个人都可以看到别人的数。
		
		每个人可以选择回答A或B,要求从小到大排序后,相邻两个人回答的不同。
	
	\end{frame}
	\begin{frame}
		\frametitle{Possiblility}
	
		有$n$个人$n$个座位,一开始第一个人会随机坐一个座位,然后其他人会依次坐在一个位子上,他们会优先选择与他们编号相同的位置,如果这个位子有人了,那么随机坐一个位置。

		求第$n$个人坐在每一个位置上的概率。
	
	\end{frame}
	\begin{frame}
		\frametitle{The Imp}
	
		有$n$个商品,价格$p_i$,价值$q_i$。

		你要依次买若干商品,买入时商品价值可能变为$0$,最多变$k$次。

		求最大的价值减价格和。

		$nk\le 10^8$。
	
	\end{frame}
	\begin{frame}
		\frametitle{Easy Equation}
	
		求$1000$组下式的正整数解,要求大小不超过$10^{100}$。
	
		$$
		a^2+b^2+c^2=k(ab+bc+ca)+1
		$$

	\end{frame}
	\begin{frame}
		\frametitle{Insider’s Information}
	
		对于排列$p$,给定若干组限制$a,b,c$,表示$p_b$是$p_a,p_b,p_c$的中位数。
		
		你需要求排列$p$,使得至少有一半限制被满足。

		特别的,至少存在一个排列使得所有限制均被满足。
	\end{frame}
	\begin{frame}
		\frametitle{Virus synthesis}
	
		给定字符串,求可以最少通过几次操作从空串变为这个字符串。
	
		\begin{enumerate}
			\item 在字符串开头或结尾加上一个字符。
			\item 翻转字符串,并加在开头或结尾。
		\end{enumerate}

	\end{frame}
	\begin{frame}
		\frametitle{Intractive}
	
		AB两人都有不超过$n$个$1\sim n$的数,在AB交流不超过$O(\log n)$bit的前提下找到中位数。
	
	\end{frame}
	\begin{frame}
		\frametitle{https://ioi2010.org/Tasks/Translations.shtml}
	
		Coming soon!
	
	\end{frame}
	\begin{frame}
		\frametitle{IOI2010 D2T4 Saveit}
	
		一个$n$个点的无权无向图,通过传输不超过70000bit信息分别表示出所有点到$1$到$m$号点的距离。
		
		$n\le 1000,m\le 36$
	\end{frame}
	\begin{frame}
		\frametitle{Parenthesis Values}
	
		定义一个括号序列的值,为每一个括号的嵌套层数的乘积。

		求所有长度$2n$的括号序列的值的和。
	
	\end{frame}
	\begin{frame}
		\frametitle{Two Coins}
	
		给定$n$,找到两个硬币,可以用来模拟一个$n$面的骰子。

		模拟必须在有限步内完成,硬币正反概率不必相同。
	
	\end{frame}
\end{document}