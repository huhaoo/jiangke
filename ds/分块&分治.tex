\documentclass[10pt]{beamer}
\usepackage{xeCJK}
\usepackage{graphicx}
\usepackage{booktabs}
\usepackage{listings}
\usepackage{multirow}
\usepackage{mathtools}
\usepackage{ulem}
\usefonttheme[onlymath]{serif}
\usetheme{metropolis}
\setbeamercolor{footnote mark}{fg=blue}% set footnote color in beamer
\setbeamerfont{footnote}{size=\tiny}
\begin{document}
	\title{分块\&分治 选讲}
	\date{\today}
	\author{huhao}
	\maketitle
	\begin{frame}
		\frametitle{前言}
	
		\sout{这次讲课范围和你们前面上的那节“数据结构的应用”有大量重叠,\\我重点就放在分块上。}

		本次讲课内容:

		序列分治$\times$

		树分治$\sqrt{}$

		序列分块$\sqrt{}$

		树分块$\times$

		本次讲课以习题为主。
	
	\end{frame}
	\section{分治}
	\begin{frame}
		\frametitle{分治}
	
		分治内容较为基础,不细讲,可以针对下面(或是其它没有写出来的)不懂的提问:

		\begin{itemize}
			\item cdq分治
			\item 二进制分组
			\item 树分治
			\item WQS二分
			\item 二分有理数
			\begin{itemize}
				\item Stern–Brocot tree
			\end{itemize}
		\end{itemize}
	
	\end{frame}
	\section{分块}
	\begin{frame}
		\frametitle{带修改序列分块}

		分块可以理解为一个三层,根度数为$w$,第二层度数为$\dfrac nw$的线段树。

	\end{frame}
	\begin{frame}
		\frametitle{续}
	
		为什么用分块而不用复杂度更加优秀的分治结构?

		\begin{itemize}
			\item 平衡时间复杂度。
			\item 便于维护信息。
			\begin{itemize}
				\item 在修改时只需要支持$\dfrac nw$个信息的合并,合并时支持$\dfrac nw$个信息的合并。
				\item 信息一共只需要经过两次处理,不需要要求信息必须可以多次相加。
			\end{itemize}
		\end{itemize}
	
	\end{frame}
	\begin{frame}
		\frametitle{例}
	
		单点修改查询前缀和,序列长度$10^8$,修改查询次数分别为:$10^8/10^4,10^6/10^6,10^4/10^8$。

		如果区间修改呢?
	
	\end{frame}
	\begin{frame}
		\frametitle{非带修序列分块}
	
		通过预处理加速询问。

		同样可以考虑一下优势区间:不方便合并大量元素产生的信息,或是存储信息花费较高等等。

		因为基于分治有一个特别优秀的做法:在线段树上每一个节点记录过中点的前/后缀和,就合并得到所有过中点的答案了。这种做法的缺点很明显:最复杂的合并是最后进行的。
	
	\end{frame}
	\begin{frame}
		\frametitle{莫队}
	
		由于分块中散块的处理需要支持多次加入单个元素,所以分块的功能会被莫队完全包含。

		但是莫队只能处理离线内容。
	
	\end{frame}
	\begin{frame}
		\frametitle{类似莫队的分块}
	
		如果某道题离线情况下可以被莫队处理,那么可以每$n^{\frac 23}$个元素分为一块,然后处理出以所有块端点作为左右端点时,莫队算法所需要的信息。
	
	\end{frame}
	\begin{frame}
		\frametitle{four russians}

		对于一个$01$序列,令$n=2^w$每$\frac wk$个元素分为一块,一共有$\dfrac{2^{w}k}{w}$块。其中一共有$\sqrt[k] n$种不同的块。

		可以尝试通过预处理每一个不同的块的信息达到加速的目的。
	
	\end{frame}
	\begin{frame}
		\frametitle{例}
	
		有以下操作$q$次:

		\begin{itemize}
			\item + x y z a b:新增一只怪物。如果这是第一只新增的怪物,那么它的编号为 1;否则它的编号为最后一只新增的怪物的编号 +1。这只怪物位于魔塔的第 x 层,它的等级为 y 级,它的难度为 z。如果玩家选择击杀这只怪物,那么需要消耗 a 点血量,在击杀成功后,玩家将得到一支可以恢复 b 点血量的药剂并立即使用。
			\item - k:删除编号为 k 的怪物。
			\item ? g l d:表示一个询问。某玩家希望击杀魔塔前 g 层中所有等级不超过 l 且难度不超过 d 的怪物。玩家可以按照任意顺序去击杀这些怪物,登上新的一层不需要杀光当前层的所有怪物,且作战过程中不会受到别的怪物的干扰。你的任务是帮助该玩家计算出征前勇士的血量最少是多少。如果某个时刻勇士的血量是负数,那么游戏结束,你一定要防止这种情况的发生。
		\end{itemize}

		请写一个程序,依次回答每个询问。注意:每个询问只是玩家的一个思考,不会真正击杀任何一只怪物。

		数据范围:$q\le 1.5\time 10^5,x,y,z\le 10^4$。
		
	\end{frame}
	\begin{frame}
		\frametitle{操作分块}
	
		在预处理后,如果一个操作对已经处理完的结果影响不大,那么可以尝试对操作进行分块,每一个块开始时进行初始化。
	
	\end{frame}
	\begin{frame}
		\frametitle{例}
	
		一颗带权有根树,支持两种操作:

		\begin{enumerate}
			\item 改变某个点的父亲,或是修改某个权值。
			\item 求某个点到它$k$级祖先中,权值大于$x$的元素个数。
		\end{enumerate}

		要求$O(n^{\frac 53})$。
	
	\end{frame}
	\begin{frame}
		\frametitle{解}
	
		每$n^\frac 13$个操作分为一块,预处理出每一个点的$n^\frac 13$级祖先,每次修改就标记一下它的$0\sim n^\frac 13$级祖先,判断能否到祖先路径是否被修改就查询一下祖先是否有标记即可。

		考虑一个点的路径,显然标记只会有$n^\frac 13$段,所有会有$n^\frac 13$次连续$n^\frac 13$步走向父亲,以及$\dfrac n{n^\frac 13}=n^\frac 23$次走向$n^\frac 13$级祖先的操作。

		总复杂度$O(n^\frac 53)$。
	
	\end{frame}
	\section{tricks}
	\begin{frame}
		\frametitle{根号分治}
	
		对于取值为$[1,n]$的变量$x$,当$x<w$和$x\ge w$时分别采用两种方法。
	
	\end{frame}
	\begin{frame}
		\frametitle{例}
	
		你一开始在$(0,0)$,你可以移动$n$次,每次移动$(1,1)$或$(1,-1)$,且不能到达$(k,-1),(k,m)$,对于每一个$i$,求到达$(n,i)$的方案数。
	
	\end{frame}
	\begin{frame}
		\frametitle{例}
	
		一个序列,需要支持区间求和,和区间每隔$k$个元素加同一个数。
	
	\end{frame}
	\begin{frame}
		\frametitle{分散层叠}
	
		对于$n$个有序序列$a_i$,可以在$O(\sum |a_i|)$的预处理后$O(n+\log \max |a_i|)$的时间复杂度内求出每一个数组中大于给定值的元素个数:

		令$b$为由以下方式生成的序列:$b_i$由$a_i$以及$b_{i-1}$偶数位归并构成(或是每$k$个取$1$个)。

		在查询$x$时,先查询$x$在$b_n$中的排位,然后就可以$O(1)$得到$b_{n-1}$中的排位,以此类推。
	
	\end{frame}
	\begin{frame}
		\frametitle{例}
	
		区间加,求区间第$k$小。
	
	\end{frame}
	\begin{frame}
		\frametitle{解}
	
		分块,分为$w$块,然后建立分散层叠的$b$数组。

		每一次修改,都会出现新的两个块,并对若干个块整体加一个数。
		
		新的块直接处理分散层叠。若干个块整体加一个元素,可以在末尾的分散层叠数组中所有元素都加上一个值,这样就可以快速还原了。

		每一次修改都会给分散层叠的数组数加$3$,不妨考虑每$w$次修改 重构一次,这样均摊单次复杂度:

		$$
		O(\dfrac 1w n+\dfrac nw+\log n(w+\log n))=O(\dfrac nw+w\log n)
		$$

		当$w=\sqrt{\dfrac n{\log n}}$时,均摊单次复杂度为$O(\sqrt{n\log n})$
	
	\end{frame}
	\section{例题}
	\begin{frame}
		\frametitle{分糖果}
	
		两个序列$a,c$,若干次操作,每一次操作会给定一个区间$[l,r]$和一个数$x$,对$i\in[l,r]$使$a_i$加$x$,并让它在$[0,c_i]$中(大于$c_i$变成$c_i$,小于$0$变成$0$)。

		求最后的$a$数组。
	
	\end{frame}
	\begin{frame}
		\frametitle{解}
	
		依次考虑$a$的每一个元素,用一个数据结构维护一下操作。

		这里维护可以考虑一下操作序列$b_1\dots b_m$,若它的一个子区间$[l,r]$和大于等于$c_i$(或小于等于$-c_i$),那么在经过$r$次操作后一定是$c_i$(或$0$)。

		然后考虑维护一下第一次取到$0/c_i$,在考虑最后一次取$0/c_i$的时候即可。
	
	\end{frame}
	\begin{frame}
		\frametitle{开店}
	
		给定一颗带权树,边有长度,多次询问:

		所有权值在$[l,r]$中的点,和$x$的距离和。

		强制在线,$n\le 1.5\times 10^5,q\le 2\times 10^5$。
	
	\end{frame}
	\begin{frame}
		\frametitle{解}
	
		考虑动态给点加权,动态询问权距离和,这就是一个经典的点分树问题。

		给点分树套用可持久化即可。
	
	\end{frame}
	\begin{frame}
		\frametitle{农民}
	
		小 D 在家种了一棵二叉树,第 $i$ 个结点的权值为 $a_i$。

		小 D 为自己种的树买了肥料,每天给树施肥。

		二叉搜索树专用版肥料是这么工作的:首先,假设所有节点权值互不相同(小 D 的二叉树可能不满足),每种权值对应一种肥料,所有肥料会从根进入树中,如果一种肥料对应的权值等于当前结点权值,这种肥料会被当前结点完全吸收,否则若肥料对应的权值小于当前结点权值,肥料会流向左子树,否则流向右子树,如果流向的子树为空,肥料只好流失蒸发了。显然,如果树是二叉搜索树,所有节点都能吸收到肥料。

		小 D 觉得自己还能抢救一下,他会进行若干次操作,每次操作修改一个点的权值或者翻转一个子树(子树内所有节点左右儿子互换)。在操作过程中,他有时会想知道一个点当前是否能吸收到肥料,以决定之后如何操作,请你帮帮可怜的小 D 吧。
	
		$n,q\le 10^5$。
	\end{frame}
	\begin{frame}
		\frametitle{解}
	
		树链剖分,记录一下从链首到链尾每一个节点肥料的上下界,以及反转后的值。

		反转标记可以类似lct地维护:访问时下传。
	
	\end{frame}
	\begin{frame}
		\frametitle{幻想乡战略游戏}
	
		给定一颗带权树,边有长度,多次询问:

		如果给第$u$号点权值加$x$,那么这个图的带权重心(距离所有点权距离和最近)是哪个点。

		每个点度数小于等于$20$,$n,q\le 10^5$。 可以考虑一下没有点度数小于$20$的条件。
	
	\end{frame}
	\begin{frame}
		\frametitle{解}
	
		在点分树上向点权和大于一半的方向移动即可。

		不难发现,可以将一个点拆成两个相连且边权为$0$的点重心不变,这样可以不断将一个点度数减小。
		
	\end{frame}
	\begin{frame}
		\frametitle{回转寿司}

		你有一个序列$a$,每一次会给出一个区间$[l,r]$并给出$A$,你需要:

		递增的对每一个$i\in[l,r]$判断:若$a_i>A$则交换$a_i,A$。

		你需要输出上述步骤后的$A$。

		$n\le 4\times 10^5,Q\le 25000$,9s。
	
	\end{frame}
	\begin{frame}
		\frametitle{解}
	
		分块,对每一块分别考虑下面两个新的问题:求操作的答案,求序列。显然后者的次数远远小于前者。

		首先,具体的答案肯定是没有问题的:考虑初始的$A$被加入$a$,最终的$A$被剔除$a$。

		其次,可以考虑记录所有的操作,假设它们形成操作序列$A_1\dots A_m$,并依次考虑块中的每一个元素$a_i$:不难发现,这是原来的问题!使用上面的办法即可做到$O(q\sqrt n\log n)$。
	\end{frame}
	\begin{frame}
		\frametitle{Innophone}
	
		有一个二元函数 $f(x,y)$,它是这么定义的:

		$$
 f(x,y)=\left\{
\begin{array}{rcl}
a, & & {\text{if} \quad \quad \ \ \ \ a \leq x}\\
b, & & {\text{else if} \quad b \leq y}\\
0, & & {\text{else}}
\end{array} \right.
$$
		其中 $a,b$ 为常数。现在给定 $n$ 组 $x,y$,你需要选择合适的 $a,b$,使得 
		$\sum_{i=1}^{n} f(x_i,y_i)$ 最大。

		$n\le 1.5\times 10^5$。
	
	\end{frame}
	\begin{frame}
		\frametitle{解}
	
		这个问题可以等效为:

		一个初始全$0$的序列$v$,每次给$x\le a$地所有$v_x$加上$x$,并查询$v$中的最大值。

		分块可以简单维护。
	
	\end{frame}
	\begin{frame}
		\frametitle{五彩斑斓的世界}
	
		一个序列,你需要:

		\begin{enumerate}
			\item 区间将大于等于$x$的元素减少$x$。
			\item 区间查询$x$出现次数。
		\end{enumerate}

		值域与序列长度同阶。

		要求$O(n^{1.5})$
	
	\end{frame}
	\begin{frame}
		\frametitle{解}
	
		分块,考虑每一块的最大值$m$,在经过一次1操作后,$m$不大于$\max(m-x,x)$。
	
		所以只要维护依次询问的代价是$O(\min(m-x,x))$的,那么无论进行多少次操作,总操作代价就是$O(m)$的。

		对$x<m-x$和$x\ge m-x$分别考虑即可。

	\end{frame}
	\begin{frame}
		\frametitle{这是我自己的发明}
	
		一颗带权树,支持若干次操作:

		\begin{enumerate}
			\item 换根。
			\item 查询两颗子树内,分别选出两个权值相同的点的方案数。
		\end{enumerate}

		要求$O(n^{1.5})$。
	
	\end{frame}
	\begin{frame}
		\frametitle{解}
	
		显然,这个问题可以变为查询两个前缀权值相同的点的方案数。

		莫队应该是最简单的做法。
		
	\end{frame}
	\begin{frame}
		\frametitle{未来日记}
	
		一个序列,你需要:

		\begin{enumerate}
			\item 区间将等于$x$的元素变为$y$。
			\item 区间查询第$k$小。
		\end{enumerate}

		要求$O(n^{1.5})$。
	
	\end{frame}
	\begin{frame}
		\frametitle{解}
	
		给序列和值域都分块,维护:

		\begin{enumerate}
			\item 每一个值在前若干序列块内出现次数。
			\item 每一个值的块在前若干序列块内出现次数。
		\end{enumerate}

		这样修改是对值域中$O(1)$乘上序列中$O(\sqrt n)$,询问是值域中$O(\sqrt n)$乘上序列中$O(1)$。
	
	\end{frame}
	\begin{frame}
		\frametitle{不归之人与望眼欲穿的人们}
	
		一个序列,你需要:

		\begin{enumerate}
			\item 单点修改。
			\item 查询$\mathrm{or}$值大于$x$的最短区间长度。
		\end{enumerate}

		$n,q\le 5\times 10^4,a\le 2^{30}$。
	
	\end{frame}
	\begin{frame}
		\frametitle{解}
	
		一个常识,一个端点固定时,不同的$\rm or$值种类数只有$30$(值域的$\log$,下记作$\log a$)种。

		分块,如果区间在一个块内,可以通过预处理的方式求解。否则,一定过块的某一个端点,枚举那个端点向左延申多少,即可做到单次$O(w\log a)$,其中$w$是块数。

		修改时重构整块,然后维护出某一块的右端点向右的$\log a$个$\rm or$值不同的区间。
	
	\end{frame}
	\begin{frame}
		\frametitle{天降之物}
	
		一个序列,你需要:

		\begin{enumerate}
			\item 将所有$x$改为$y$。
			\item 查询最靠近的$x$和$y$的距离。
		\end{enumerate}

		$n,q\le 10^5$。强制在线。
	
	\end{frame}
	\begin{frame}
		\frametitle{解}
	
		如果询问的$x,y$出现次数均小于$\sqrt n$次,将它们出现位置合并即可。

		否则可以$x,y$有一个出现次数大于$\sqrt n$次,可以预处理出所有这种情况的答案。

		考虑修改中合并的情况:

		如果$x,y$出现次数均不大于$\sqrt n$,或均大于$\sqrt n$,暴力修改即可。

		否则将它们均保留下来,查询时分别查询,合并时就合并在出现次数不大于$\sqrt n$的上面,如果合并后大于$\sqrt n$就全部合并起来。
	
	\end{frame}
	\begin{frame}
		\frametitle{D2T2}
	
		你有一个序列,有若干询问:

		在区间$[l,r]$中,仅保留$[L,R]$的数,最大区间子段和是多少。

		$n\le 10^5$。
	
	\end{frame}
	\begin{frame}
		\frametitle{解}
	
		考虑分成$\sqrt n\times \sqrt n$的块,每一块都只有$\sqrt n^2$种不同的$[L,R]$的答案。

		维护一段的信息可以分治求解:$T(n)=2T(\frac n2)+O(n^2)=O(n^2)$(考虑$\sum_i 2^i(\dfrac{n}{2^i})^2$)。
	
	\end{frame}
\end{document}