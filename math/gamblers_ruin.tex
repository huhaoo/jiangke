\documentclass[10pt]{beamer}
\usepackage{xeCJK}
\usepackage{graphicx}
\usepackage{booktabs}
\usepackage{listings}
\usepackage{multirow}
\usepackage{mathtools}
\usepackage{ulem}
\usepackage{amsmath}
\usefonttheme[onlymath]{serif}
\usetheme{metropolis}
\setbeamercolor{footnote mark}{fg=blue}% set footnote color in beamer
\setbeamerfont{footnote}{size=\tiny}
\begin{document}
	\title{赌徒破产问题}
	\date{\today}
	\author{huhao}
	\maketitle
	\clearpage
	\begin{frame}
		\frametitle{}
		Consider a game that gives a probability α of winning 1 dollar and a probability β = 1–α of losing 1 dollar. If a player begins with (say) 10 dollars, and intends to play the game repeatedly until he either goes broke or increases his holdings to N dollars, what is his probability of reaching his desired goal before going broke.\footnote{www.mathpages.com/}
	\end{frame}
	\begin{frame}
		\frametitle{}
	
		考虑一组i.i.d随机变量$X_1\dots X_n$,其中$P(\nu\le X\le \mu)=1$,且$P(X=\nu)>0,P(X=\mu)>0$,令$S_n=\sum_{i\le n}X_n,T=\min\{n\ge 1:S_n\le -L\vee S_n\ge H\}$,求:

		$$
		P(S_{T}\le -L)
		$$

		在下面,一般只考虑$X\in Z$的情况。
	
	\end{frame}
	\begin{frame}
		\frametitle{例}
	
		$P(X=-1)=p,P(X=1)=q,P(X=0)=1-p-q$,求:

		$$
		P(S_T=1),ET
		$$

		如果$P(X=-2)=p$呢
	
	\end{frame}
	\begin{frame}
		\frametitle{}
	
		不妨令$u(i)$为$S_0=i$时(即令$S_n=i+\sum X_j$),$P(S_T\ge H)$的值。

		不难发现:

		\begin{equation}
			u(i)=\sum_{j}P(X=j)u(i+j) \tag{1}
		\end{equation}

		以及
		
		\begin{equation}
			u(i)=0,i\le -L\tag{2}
		\end{equation}
		\begin{equation}
			u(i)=0,i\ge H \tag{3}
		\end{equation}
	
	\end{frame}
	\begin{frame}
		\frametitle{}
	
		不妨考虑$1=\sum_i x^iP(X=i)$,显然它的解和:

		$$
		f(x)=x^\nu-\sum_i x^{i+\nu}P(X=i)
		$$

		零点相同,不妨令$\lambda_i$为这些零点,且$k_i$为重数,则:

		$$
		u(x)=\lambda_i^x
		$$

		显然可以满足(1)。
	
	\end{frame}
	\begin{frame}
		\frametitle{}
	
		更为一般的,若$f(x)=(x-\lambda_i)^j g(x)$,则有:

		$$
		u(x)=x^j\lambda_i^x
		$$

		也满足(1)。

	
	\end{frame}
	\begin{frame}
		\frametitle{}
	
		令

		$$
		u(x)=\sum_{i,j} a_{i,j}x^j\lambda_i^x
		$$

		那么显然$u$满足(1)。

		又因为$a$一共有$\nu+\mu$个变元,而边界条件可以看作又$\nu+\mu$个等式,所以可以通过解出$a_{i,j}$求出$u$的表达式。
	
	\end{frame}
	\begin{frame}
		\frametitle{例}
	
		$P(X=2)=p,P(X=-1)=q$,解$u$。
	
	\end{frame}
	\begin{frame}
		\frametitle{}
	
		类似的,可以发现,若要求$EN$,则有:

		$$
		v(i)=1+\sum_j P(X=j) v(i+j)
		$$

		于是$v=v_g+v_p$,$v_g$是齐次方程的解,$v_p$是任一特解。

		如果$EX\not=0$,那么$v_p(x)=cx$能解出一特解。(否则可以试试$v_p=cx^2$,再否则试试$v_p=cx^3$……)
	
	\end{frame}
	\begin{frame}
		\frametitle{}
	
		再来考虑一下$H=+\infty$的情况,一般来说,只需要使用上面的方法,令$H\rightarrow \infty$即可。

		但是,有一种更为简便的方法。

		当然,如果$EX\le 0$,那么显然赌徒一定会破产。
	
	\end{frame}
	\begin{frame}
		\frametitle{}
	
		考虑之前的那个$f$,\sout{显然}它在$|z|<1$有$\nu$个零点$\eta_1\dots \eta_\nu$。

		则:

		$$
		P_{ruin}(M)=\sum_{n=1}^\nu \Phi_{n,M-n+1}(\eta_1\dots \eta_n)\prod_{j=1}^{n-1}(1-\eta_j)
		$$

		其中$\Phi$是完全齐次对称多项式:

		$$
		\Phi_{n,r}(z_1\dots z_n)=\sum_{i_j\ge 0,\sum i_j=r}\prod_{j=1}^nz_j^{i_j}
		$$

		若$\eta$互不相同,则:

		$$
		P_{ruin}(M)=\sum_{j=1}^{\nu}\eta_j^M\prod_{i\not=j}\dfrac{1-\eta_i}{\eta_j-\eta_i}
		$$
	
	\end{frame}
	\begin{frame}
		\frametitle{例}
	
		$P(X=-2)=P(X=1)=\dfrac 12$

		\onslide<2->

		$$
		1=\dfrac{1}{2}(x+x^{-2})
		$$

		解得:

		$$
		\{x\to 0.877439\, -0.744862 i\},\{x\to 0.877439\, +0.744862 i\},\{x\to -0.754878\}\
		$$

		\sout{所以答案是$-0.754878^M$。}

	\end{frame}
	\begin{frame}
		\frametitle{例}
	
		$P(X=-2)=P(X=3)=\dfrac 12$

		\onslide<2->

		不难发现:

		$$
		\{\{x\to 0.848375\},\{x\to -0.660993\}\}
		$$

		直接代回那个式子即可。
	
	\end{frame}
	\begin{frame}
		\frametitle{St. Petersburg Lottery}
	
		$P(X=-m+2^k)=2^{k-1},k\in Z$,求$m=15$时:

		$$
		P_{ruin}(10^9)
		$$

		保留$7$为小数。
	
	\end{frame}
\end{document}