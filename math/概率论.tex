\documentclass[10pt]{beamer}
\usepackage{xeCJK}
\usepackage{graphicx}
\usepackage{booktabs}
\usepackage{listings}
\usepackage{multirow}
\usepackage{mathtools}
\usepackage{ulem}
\usefonttheme[onlymath]{serif}
\usetheme{metropolis}
\setbeamercolor{footnote mark}{fg=blue}% set footnote color in beamer
\setbeamerfont{footnote}{size=\tiny}
\begin{document}
	\title{概率论选讲}
	\date{\today}
	\author{huhao}
	\maketitle
	\clearpage
	\begin{frame}
		\frametitle{测度论}
	
		\sout{这阴间玩意要讲吗。}

		\sout{这阴间玩意的PPT是人写的吗。}
	
	\end{frame}
	\begin{frame}
		\frametitle{重要等式}
	
		若$X_1\dots X_n$是独立的,则:

		$$
		\mathrm{var}(\sum X_i)=\sum \mathrm{var}X_i
		$$
	
	\end{frame}
	\begin{frame}
		\frametitle{大数定理}
	
		若$X_1\dots $有相同分布且是独立的(简称i.i.d.),且$E|X_1|<\infty$,则:

		$$
		\dfrac 1n\sum_{i=1}^n X_i\rightarrow EX_1
		$$

		\sout{很符合直觉}
	
	\end{frame}
	\begin{frame}
		\frametitle{例}
	
		$X_1\dots $是i.i.d.,且$P(X_1=2^{-n}-k)=2^{-n},n\in Z^+$,估计:

		$$
		n^{-1}\sum_{i=1}^n X_i
		$$

		\onslide<2->

		显然会趋近于$\infty$,但是上面的定理不能直接用,可以通过删去$X>2^{k+2}$部分来使用定理。

		当然有更加精确的估计:

		$$
		\dfrac 1{n\log_2 n}\sum_{i=1}^n X_i\rightarrow 1
		$$
	
	\end{frame}
	\begin{frame}
		\frametitle{切比雪夫不等式}
	
		对于任意集合$A$和函数$\phi$,有:

		$$
		P(X\in A) \inf_{y\in A} \phi(y)\le E \phi(X)
		$$

		最为常用的特例是:$a^2P(|X|\ge a)\le EX^2$。
	
	\end{frame}
	\begin{frame}
		\frametitle{例}
	
		连续投一个硬币$n$次,估计投出正面次数的范围。

		\onslide<2->

		定义$X_i$根据第$i$次投出的正反决定是$+1/-1$,那么:

		$$
		EX_i^2=1
		$$

		所以:

		$$
		P(|\sum X|\ge m)\le m^2n
		$$

		若要它小于$\epsilon$,令$m=\sqrt{n\epsilon^{-1}}$即可。

		所以可以发现投出正面的次数几乎都在以$\dfrac n2$为中心$O(\sqrt n)$的区间内。
	
	\end{frame}
	\begin{frame}
		\frametitle{例}
	
		你若通过蒙特卡洛算法去求一个$\mathrm{var}X=\sigma^2$的$X$的期望,那么你运行$n$次,可以得到:

		$$
		P(|\dfrac 1n\sum X-EX|\ge m)\le m^2n\sigma^2
		$$

		所以每多运行$100$倍的$n$你就能多信任一位的数字。

		(这是一个比较松的下界,但是中心极限定理会告诉你上面那句话不能再紧了)
	
	\end{frame}
	\begin{frame}
		\frametitle{Kolmogorov’s maximal inequality}
	
		对于$X_1\dots $,它们相互独立且$EX_i=0$,则:

		$$
		P(\max_i |S_i|\ge x)\le x^{-2}\mathrm{var} S_i
		$$

	\end{frame}
	\begin{frame}
		\frametitle{中心极限定理}
	
		若$X_1\dots $是i.i.d.,且$EX_1=\mu, \mathrm{var}X_1=\sigma^2\in(0,\infty)$,则:

		$$
		\dfrac{1}{\sigma\sqrt n}\sum_{i=1}^n(X_i-\mu)\Rightarrow \chi 
		$$

		这个渐进式是不是比切比雪夫不等式的特例强很多?

		\onslide<2->

		\sout{你敢直接认为$a_i=\lim_{n\rightarrow \infty} a_n$吗。}

		例题可以看SDOI2017 龙与地下城。
	
	\end{frame}

\end{document}