\documentclass[10pt]{beamer}
\usepackage{xeCJK}
\usepackage{graphicx}
\usepackage{booktabs}
\usepackage{listings}
\usepackage{multirow}
\usepackage{mathtools}
\usepackage{ulem}
\usefonttheme[onlymath]{serif}
\usetheme{metropolis}
\setbeamercolor{footnote mark}{fg=blue}% set footnote color in beamer
\setbeamerfont{footnote}{size=\tiny}
\begin{document}
	\title{模意义下乘法群性质}
	\date{\today}
	\author{huhao}
	\maketitle
	\clearpage
	\begin{frame}
		\frametitle{定义}
	
		\begin{itemize}
			\item 代数系统$(A,O_1,O_2,\dots,O_m)$:若干元素组成的集合$A$,以及若干元素间的运算$O_i$,满足定义域是$A^k$,值域是$A$。
			\item 半群$(A,\cdot)$:代数系统;$\cdot:A^2\rightarrow A$,且$x(yz)=(xy)z$。
			\item 群的积:$(A,\cdot)\times(B,\cdot)=\{A\times B,\cdot\}$,且$(u,v)\cdot(a,b)=(u\cdot a,v\cdot b)$。
			\item 含幺半群$(A,\cdot)$:半群;$\exists e,xe=x$。可以发现,$ex=x$。
			\item 群$(A,\cdot)$:含幺半群;$\forall x,\exists y,yx=e$。可以证明,只要任意一个元素存在左逆元,那么它就有唯一的逆元。
			\item Abel群$(A,\cdot)$:群;$xy=yx$。 Abel群一个很重要的性质是$a^kb^k=(ab)^k$。
			\item 循环群$(A,\cdot)$:群;$\exists x,\forall y,\exists k,x=y^k$,不难发现,循环群是Abel群。
			\item 置换群$(A,\cdot)$:$A$中所有元素均为函数,由$1\sim n$的排列$p$所确定:$f_p(i)=p_i$,且若$xy=z$则$f_y(f_x(i))=f_z(i)$,不难发现,这是群。
		\end{itemize}

	\end{frame}
	\begin{frame}
		\frametitle{同构与同态}
	
		若$f:A\rightarrow B$,$(A,\cdot),(B,\cdot)$是代数系统,且$f(a)+f(b)=f(ab)$则称($f$是)$A,B$(的)同态。

		如果$f$是双射,则称为同构。

		容易验证,所有$n$维循环群是模$n$意义下加法群(下面简记为$Z_n$)的同构。

		容易证明,同态与同构会保留上一页中代数系统的性质。
	
	\end{frame}
	\begin{frame}
		\frametitle{生成元}
	
		在群中, 定义$\langle a\rangle =\{x|x=a^k\}$,$O\langle a\rangle =|\langle a\rangle |$称为$a$的阶,若$O\langle x\rangle =|A|$,则称$x$为$A$的生成元。

		不难发现。
		
		\begin{itemize}
			\item $x\in Z_n,O\langle x\rangle |n$。
			\item $O\langle x\rangle =O\langle -x\rangle $。
		\end{itemize}
	\end{frame}
	\begin{frame}
		\frametitle{Abel群的性质}
	
		\textbf{所有有限Abel群$(A,\cdot)$与$Z_{i_1}\times Z_{i_2}\times \dots Z_{i_n}$(若干循环群的积)同构。}
		
		记$|A|=n$,取$O\langle x\rangle $最大的$x$,则$\langle x\rangle $构成子群,且可以找出$n/O\langle x\rangle $个大小为$O\langle x\rangle $集合,使得集合中所有元素乘上$\langle x\rangle $中的元素依然在这个集合中(这可以不断找一个元素$u$,然后生成集合$\{u^k\}$,这样不同集合肯定是不交的)。

		若$O\langle x\rangle \not=n$,则有多个集合,若将这些集合间的乘运算规定为各取一个元素相乘,得到的结果在的集合。则这些集合和乘法也构成群$(B,\cdot)$。

		假如$(B,\cdot)$与$C=Z_{i_1}\times Z_{i_2}\times \dots Z_{i_n}$同构,若能证明$A$与$C\times Z_{O\langle x\rangle }$同构,则可根据归纳法证明。

		不妨把$B$中的元素(一个集合),通过同构的函数映射至$C$中的元素,通过$(j_1,j_2,\dots j_n),0\le j_k<i_k$表示。
	\end{frame}
	\begin{frame}
		\frametitle{续}
	
		对于群乘积的每一维$Z_{i_k}$,可以在$(0,0,\dots,1,\dots 0)$对应集合$t_k$中(第$k$位为$1$)(暂时)任取一个元素$u_k$作为这个群的代表,假定$u_k^{i_k}=e_A$(否则下面将在$t_k$中找到一个代替这个$u_k$的元素),因为后续证明需要用到这一点。

		如果$u_k^{i_k}\not=e_A$,那么有$u_k^{i_k}=x^y$,则(下式用到了$u_k^r=e_A\Rightarrow i_k|r$,这个可以是根据$t_k$的定义得到的):
		
		$$
			O\langle u_kx^l\rangle =i_k\dfrac{O\langle x\rangle }{(y+li_k,O\langle x\rangle )}
		$$

		根据$x$的定义,有:$i_k\le(y+li_k,O\langle x\rangle )$

		取$l=\lfloor\dfrac{y}{i_k}\rfloor$即可得到$i_k|y$。

		则$u_k'=u_kx^{-\frac y{i_k}}$满足$u_k'^{i_k}=e_A$,用$u_k'$代替任取的$u_k$即可。
	
	\end{frame}
	\begin{frame}
		\frametitle{续}
	
		对于集合$(j_1,\dots j_n)$,不妨用$r_{j_1,\dots j_n}=\prod_{k}u_k^{j_k}$代表(根据定义,则个元素一定在集合内)。

		所以$(\{r\},\cdot)$对运算封闭,它是一个群。可以将$A$中的元素表示为$(u,v),u\in \{r\},v\in \langle x\rangle $,这样可以唯一的表示出$|A|=|\{r\}|O\langle x\rangle $个元素,如果将这样的一一对应关系记为函数$f$,则$f$是$(A,\cdot)$与$(\{r\},\cdot)\times(\langle x\rangle ,\cdot)$的同构。

		所以$A$与$C\times Z_{O\langle x\rangle }$同构(这里没有证明$\{r\}$与$C$同构,可以尝试自行证明。同时可以发现此部分可以不证明,因为已经可以归纳地找到与$r$同构的若干循环群的积了),证明完毕。
	
	\end{frame}
	\begin{frame}
		\frametitle{续}
	
		更进一步,我们知道了$A$与$\prod_j Z_{i_j}$同构,则有:$|Z_{i_j}|~{\big|}~|A|$。

		则对于$A$中的元素$x$,通过同构映射至$\prod_j Z_{i_j}$中的元素$(x_1\dots x_k)$,则$x^{|A|}$可以映射至$(x_1\dots x_k)^{|A|}=(x_1^{|A|}\dots x_k^{|A|})=(e_1\dots e_k)$,即$x^{|A|}=e$。
	
	\end{frame}
	\begin{frame}
		\frametitle{将Abel群分解为循环群}
	
		和证明中的构造方式一致:

		不妨记要分解的Abel群为$A$,遍历$A$中元素,找到$O\langle x\rangle$最大的$x$,提出所有$x^k$。

		然后将$A$划分为$\dfrac{|A|}{O\langle x\rangle}$个集合,递归的将这些集合划分为循环群。

		然后找出每个集合的代表元,这些代表元组成集合$B$,就将$A$划分为了$B\times \langle x\rangle$,$B$划分为循环群方式在上一步已经计算出来了。

		这样每一次都会使要划分为循环群的元素个数除以$2$,这样就可以$O(n\log n)$的时间复杂度内划分开。
	
	\end{frame}
	\begin{frame}
		\frametitle{循环群性质}
	
		\textbf{$Z_{xy}$与$Z_x\times Z_y$同构,其中$(x,y)=1$}。
		
		如果令$1_A$为$A$的生成元,则:

		\begin{itemize}
			\item $1_{xy}=(1_x,1_y)$
			\item $(1_x,1_y)=(1_{xy}^y,1_{xy}^x)$
		\end{itemize}

		上面分别给出了双向的构造。

		\textbf{循环群$Z_n$生成元个数为$\varphi(n)$。}

		考虑一个生成元$g$,则可以将其它元素写成$g^k$的形式,$O\langle g^k\rangle=\dfrac{n}{(n,k)}$,则满足$(n,k)=1$的$g^k$是生成元,则一共是$\varphi(n)$个。
	
	\end{frame}
	\begin{frame}
		\frametitle{模意义下乘法(半)群}
	
		对于模$n$意义下的乘法半群,如果仅考虑$Z_n^*=\{x|(n,x)=1\}$,那么这就是一个Able群(可以通过裴蜀定理证明),可以用$\varphi(n)$来表示出它的元素个数。

		由Abel群的性质可知:$(n,x)=1\Rightarrow x^{\varphi(n)}=1$。
	
	\end{frame}
	\begin{frame}
		\frametitle{$Z_p^*$}
	
		对于奇质数$p$,$Z_p^*=([1,p-1],\times)$是Abel群。在这个群中,$f(x)=a_nx^n+\dots+a_0=0$的解的个数不超过$n$:

		若$x_n$是方程的解,那么$f(x)=(x-x_n)g(x)$,而$g(x)$最多有$n-1$个解不在这$1+(n-1)$个解中的元素$y$这会使$(y-x_n)$和$g(y)$都不是$0$。

		根据这个定理(拉格朗日定理),可以证明上述群是循环群:

		根据Abel群与若干循环群之积同构,可以得到:对于任意$x\in[1,p-1]$,有$O\langle x\rangle|p-1$,因为$O\langle x\rangle$等于在每一个循环群上阶的最小公倍数。

		所以不妨令$S_d=\{x|O\langle x\rangle=d\}$,于是若$d$不是$p-1$的约数,一定有$S_d=\emptyset$。

		又如果{\color{red}有}$O\langle x\rangle=d$,那么$x^{k}=1$就是$x^d=1$的解,根据拉格朗日定理,这$d$个数就是唯$d$解,如果$(k,d)=1$,那么$O\langle x^k\rangle=d$,即$|S_d|=\varphi(d)$。
	
	\end{frame}
	\begin{frame}
		\frametitle{续}
	
		于是有$S_d=0$或$S_d=\varphi(d)$(前者是上一页中把红色的字换成“没有”的情况),又有:

		$$
		p-1=\sum_i |S_i|\le \sum_{d|p-1} \varphi(d)=p-1
		$$

		所以$|S_d|=\varphi(d)[d|p-1]$。所以$S_{p-1}\not=\emptyset$,于是就证明了上面讨论的群是循环群。称原根为满足$O\langle x\rangle=p-1$的$x$,即这个群的生成元。
	
	\end{frame}
	\begin{frame}
		\frametitle{$Z^*_{p^2}$}
	
		不妨再看看模$p^2$下的情况,现在我们要证明的是存在$x\in Z^*_{p^2},O\langle x\rangle=(p-1)p=\varphi(p^2)$。

		不妨考虑$p$的原根$g$,对于$g+ip,i\in Z^+$,有:$\varphi(p)|O\langle g+ip\rangle,O\langle g+ip\rangle|\varphi(p^2)$,所以$O\langle g+ip\rangle\in\{\varphi(p),\varphi(p^2)\}$。

		假如$Z_{p^2}^*$不是循环群,则$O\langle g\rangle=O\langle g+p\rangle=\varphi(p)$,则:

		$$
		\begin{aligned}
			&1=(g+p)^{\phi(p)}=\sum_{i=0}^{p-1}{p-1\choose i}p^{i}g^{p-1-i}=\\
			&1+g^{-1}p(p-1)+0+\dots+0=1-g^{-1}p
		\end{aligned}
		$$

		则$p|g^{-1}$,矛盾,所以$g,g+p$一定有一个阶为$\varphi(p^2)$。
	
	\end{frame}
	\begin{frame}
		\frametitle{$Z^*_{p^c}$}
	
		考虑$g$为$Z^*_{p^2}$的原根,则可以通过归纳法:假设$g$为$Z^*_{p^{c-1}}$的原根,则在$Z^*_{p^c}$中有$O\langle g\rangle\in\{\varphi(p^{c-1}),\varphi(p^c)\}$,不妨设:

		$$
		g^{\varphi(p^{c-2})}=g^{(p-1)p^{c-3}}=1+p^{c-2}k\not=1
		$$

		则:

		$$
		g^{\varphi(p^{c-1})}=(1+p^{c-2}k)^p=1+p^{c-1}k
		$$

		由于$k\not=0$,也就是说$g^{\varphi(p^{c-1})}\not=1$,所以$O\langle g\rangle=\varphi(p^c)$
	
	\end{frame}
	\begin{frame}
		\frametitle{$Z^*_{2p^c}$}
	
		类似的,考虑一下$Z^*_{2p^c}$,不难证明:$f(x)=x\bmod p^c$是$Z^*_{2p^c}$到$Z^*_{p^c}$的同构映射,所以它的性质和$Z^*_{p^c}$一样。
	
	\end{frame}
	\begin{frame}
		\frametitle{总结}
	
		对于奇质数$p$,$Z^*_{p^c},Z^*_{2p^c}$都是循环群,即有原根,且恰有$\varphi(\varphi(p^c))$个。

		不难验证,$Z^*_2,Z^*_4$都有原根,分别是$1;1,3$。

		上述群中元素可以用一个正整数来代表,乘法操作就可以变为$Z_{\varphi(p^c)}$上的加操作。

		对于其它的整数$n$,$Z^*_n$是Abel群,可以划分为若干循环群之积。群中元素可以用一个数组代表,乘法操作也可以变为若干循环群积上的加操作。
	
	\end{frame}
	\begin{frame}
		\frametitle{例}
	
		给定$n,a,b$,满足$n\le 10^6,(b,n)=1$,求下式解的个数:

		$$
		x^a \bmod n=b
		$$

		加强版:没有$(b,n)=1$的限制。
	
	\end{frame}
\end{document}