\documentclass[10pt]{beamer}
\usepackage{xeCJK}
\usepackage{graphicx}
\usepackage{booktabs}
\usepackage{listings}
\usepackage{multirow}
\usepackage{mathtools}
\usepackage{ulem}
\usefonttheme[onlymath]{serif}
\usetheme{metropolis}
\setbeamercolor{footnote mark}{fg=blue}% set footnote color in beamer
\setbeamerfont{footnote}{size=\tiny}
\begin{document}
	\title{组合数学选讲}
	\date{\today}
	\author{huhao}
	\maketitle
	\clearpage
	\begin{frame}
		\frametitle{一些组合恒等式}

		$$
		\begin{aligned}
			&{n+m\choose n}={n+m\choose m}\\
			&{n\choose m}=\dfrac nm{n-1\choose m-1}\\
			&{n\choose m}{m\choose k}={n\choose n-m,m-k,k}={n\choose k}{n-k\choose m-k}\\
			&\sum_{i=l}^r{i\choose m}={r+1\choose m+1}-{l\choose m+1}\\
			&\sum_{i}{n\choose i}{m\choose k-i}={n+m\choose k}\\
		\end{aligned}
		$$

	\end{frame}
	\begin{frame}
		\frametitle{组合数行求和}

		二项式定理:

		$$
		f_n(x)=(1+x)^n=\sum_i {n\choose i}x^i
		$$
		
		带入$x=1$可以知道:

		$$
		f_n(1)=2^n=\sum_i{n\choose i}
		$$

	\end{frame}
	\begin{frame}
		\frametitle{续}
	
		带入$\omega_k=\cos \dfrac{2\pi}{k}+i\sin \dfrac{2\pi}{k}$的$0\sim k-1$次幂:

		$$
		f(\omega_k^j)=(1+\omega_k^j)^n=\sum_i {n\choose i}\omega_k^{ij}
		$$

		对上式所有$j$求和:

		$$
		\sum_j f(\omega_k^j)=\sum_{i}\sum_j{n\choose i}\omega_k^{ij}=\sum_i {n\choose i}[k|i]k
		$$
	
	\end{frame}
	\begin{frame}
		\frametitle{续}
	
		更一般的:

		$$
		\sum_j \omega_k^{jt}f(\omega_k^j)=\sum_{i}\sum_j{n\choose i}\omega_k^{(i+t)j}=\sum_i {n\choose i}[k|i+t]k
		$$

		更更一般的,这个方法对于所有收敛半径大于$1$的幂级数都是有效的:

		$$
		\dfrac 1k\sum_j \omega_k^{tj}g(\omega_k^j)=\sum_{i}g_i[k|i+t]
		$$
	
	\end{frame}
	\begin{frame}
		\frametitle{反演}
	
		如果有$f=Ag$那么就有$g=A^{-1}f$。

		例如$A_{x,y}={x\choose y}$因为:

		$$
		\sum_{i=0}^{n-1} {u\choose i}{i\choose v}(-1)^{i-v}=[u=v]
		$$

		所以$A^{-1}_{x,y}={x\choose y}(-1)^{x-y}$,那么就有:

		$$
		f_n=\sum_i {n\choose i} g_i\Leftrightarrow g_n=\sum_i (-1)^{n-i}{n\choose i}f_i
		$$
	
	\end{frame}
	\begin{frame}
		\frametitle{续}
	
		再例如$A_{x,y}=[x\in y],A_{x,y}=[x|y],A_{x,y}=x-y\dots$
	
	\end{frame}
	\begin{frame}
		\frametitle{斯特林数}
	
		第二类斯特林数${n\brace m}$表示将$n$个不同的元素放入$m$个相同的集合中的方案数。

		第一类斯特林数${n\brack m}$表示将$1\sim n$的轮换数为$m$的方案数。

		根据定义,可以得到递推式:

		$$
		\begin{aligned}
			{n\brace m}={n-1\brace m-1}+m{n-1\brace m}\\
			{n\brack m}={n-1\brack m-1}+(n-1){n-1\brack m}
		\end{aligned}
		$$
		
	\end{frame}
	\begin{frame}
		\frametitle{续}
	
		不难发现:

		$$
		\begin{aligned}
			x^{\overline n}=&\sum_i x^i{n\brack i}\\
			x^{\underline n}=&\sum_i x^i{n\brack i}(-1)^{n-i}\\
			x^n=&\sum_i x^{\underline i}{n\brace i}
		\end{aligned}
		$$

		上两式的可以根据递推式得到,下面的可以用给$n$个不同的物体染$x$种颜色推出。
	
	\end{frame}
	\begin{frame}
		\frametitle{续}
	
		又由于:

		$$
		x^n=\sum_i {n\brace i} \sum_{m} {i\brack m} x^m (-1)^{i-m}
		$$

		固定$n,m$,则有:

		$$
		[n=m]=\sum_i{n\brace i}{i\brack m}(-1)^{i-m}
		$$
	
	\end{frame}
	\begin{frame}
		\frametitle{续}
	
		不难发现:

		$$
		\begin{aligned}
		&\sum_i {n\brace i}(-1)^{n-i}x^{\overline{i}}\\
		=&\sum_i {n\brace i}(-1)^{n-i}\sum_j x^j{i\brack j}\\
		=&\sum_{i,j} (-1)^{n-i}x^j{n\brace i}{i\brack j}\\
		=&x^n
		\end{aligned}
		$$
	
	\end{frame}
	\begin{frame}
		\frametitle{续}
	
		若记$A_{i,j}={i\brace j}$,那么$A_{i,j}^{-1}={i\brack j}(-1)^{i-j}$,$A$所对应的就是斯特林反演。
	
	\end{frame}
	\section{习题}
	\begin{frame}
		\frametitle{2018雅礼集训 方阵}
	
		给定 $n \times m$ 的矩阵,每个格子填上 $[1, c]$ 中的数字,求任意两行、两列均不同的方案数。
		$n, m \le 5000$。
	
	\end{frame}
	\begin{frame}
		\frametitle{HNOI2019 白兔之舞}
	
		给定矩阵$W$和正整数$L,k$,对于$t\in[0,k)$求:

		$$
		\sum_{i}[k|i+t]W^i{L\choose i}
		$$

		$k\le 10^5$
	
	\end{frame}
	\begin{frame}
		\frametitle{联合省选2020 A卷 组合数问题}
	
		计算 
		$$\left(\sum_{k=0}^{n}f(k)\times x^k\times \binom{n}{k}\right)\bmod p$$ 
		的值。其中 $n$, $x$, $p$ 为给定的整数,$f(k)$ 为给定的一个 $m$ 次多项式 $f(k) = a_0 + a_1k + a_2k^2 + \cdots + a_mk^m$。

		$1\le n, x, p \le 10^9, 0\le a_i\le 10^9, 0\le m \le \min(n,1000)$。
	\end{frame}
	\begin{frame}
		\frametitle{思考题}
	
		$$
		\sum_{k}\dfrac{(-1)^k}{k+1}{n+k\choose 2k}{2k\choose k}
		$$
	
		$$
		\sum_i {m\choose 2i}{m-2i\choose n-i}2^{2i}
		$$
	
	\end{frame}
\end{document}