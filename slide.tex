\documentclass[10pt]{beamer}
\usepackage{xeCJK}
\usepackage{graphicx}
\usepackage{booktabs}
\usepackage{listings}
\usepackage{multirow}
\usepackage{mathtools}
\usepackage{ulem}
\usefonttheme[onlymath]{serif}
\usetheme{metropolis}
\begin{document}
	\title{solution}
	\date{\today}
	\author{huhao}
	\maketitle
	\clearpage
	\begin{frame}
		\frametitle{前言}
		本场考试考察知识点如下:

		\begin{enumerate}
			\item \textbf{概率与期望,DP},{\color{gray}数据结构,矩阵,概率生成函数}。
			\item \textbf{DP,计数},生成函数。
			\item \textbf{组合恒等式,组合意义},lucas定理,{\color{gray}生成函数}。
			\item \textbf{构造},数论。
		\end{enumerate}

		几乎覆盖了NOIP的所有{\color[rgb]{0.8,0.8,0.8}数学}知识点。
	\end{frame}
	\clearpage
	\begin{frame}
		\frametitle{qiu}

		\onslide<1-> 概率生成函数似乎很好做,但是求导似乎不太友善。

		\onslide<2-> 不如DP,记$p_{i,j}$表示第$i$个酒桶,距今年份$j$年的酒量,记:
		$$
		\begin{aligned}
			f_i=&\sum_{j}p_{i,j}\\
			g_i=&\sum_{j}p_{i,j}j\\
			h_i=&\sum_{j}p_{i,j}j^2\\
		\end{aligned}
		$$
		\onslide<3-> 则
		$$
		\begin{aligned}
		f_i=&f_{i-1}a_{i-1}+f_i(1-a_i)\\
		g_i=&a_{i-1}(g_{i-1}+f_{i-1})+(1-a_i)(g_i+f_i)\\
		h_i=&a_{i-1}(h_{i-1}+2g_{i-1}+f_{i-1})+(1-a_i)(h_i+2g_i+f_i)
		\end{aligned}
		$$
		\onslide<4-> 线段树维护转移矩阵,复杂度$O(n\log n)$。
	\end{frame}
	\clearpage
	\begin{frame}
		\frametitle{qiu}
		\onslide<1->化简:
		$$
		\begin{aligned}
			a_if_i=&a_{i-1}f_{i-1}\\
			a_ig_i=&a_{i-1}g_{i-1}+f_i\\
			a_ih_i=&a_{i-1}h_{i-1}+2g_i-f_i
		\end{aligned}
		$$

		\onslide<2->可以发现
		$$
		\begin{aligned}
			f_i=&\dfrac{1}{a_i}\\
			g_i=&\dfrac{1}{a_i}\sum_j f_i\\
			h_i=&\dfrac{1}{a_i}\sum_j 2g_j-f_j
		\end{aligned}
		$$

	\end{frame}
	\clearpage
	\begin{frame}
		\frametitle{qiu}
	
		\onslide<1-> 发现答案只和$a_nf_n,a_ng_n,a_nh_n$有关,有:
		$$
		\begin{aligned}
			a_nf_n&=1\\
			a_ng_n&=\sum_i \dfrac{1}{a_i}\\
			a_nh_n&=\sum_{i\le j}\dfrac{2}{a_ia_j}-\sum_{i} \dfrac{1}{a_i}
		\end{aligned}
		$$

		\onslide<2-> 维护$\sum_i\frac{1}{a_i},\sum_i\frac{1}{a_i^2}$即可,运用$O(\bmod)$求逆即可总复杂度$O(n+q+\bmod)$求解。
	
	\end{frame}
	\clearpage
	\begin{frame}
		\frametitle{xxx}
	
		\onslide<1-> 事实上,可以发现,可以将那张图每个点所代表的视为$[1,n]$的数:
		\begin{itemize}
			\item 对于数$i$,若存在$A$,使得$i\in A\in S$,那么$i$在图中。
			\item 对于数$i,j$,若存在$A$,使得$i\in A,j\in A$,那么$i,j$有边。
			\item $f(S)$同样也是这张图的联通块数。
		\end{itemize}
		
		\onslide<2-> 简单容斥即可,复杂度$O(n^2k)$。{\color{gray}多项式快速幂可以$O(n\log n)$}。
	\end{frame}
	\clearpage
	\begin{frame}
		\frametitle{final}
		\onslide<1->可以发现,原式等于:
		$$
		\sum_i\dfrac{2^{2i+1}(m+1)!}{(2i+1)!(n-i)!(m-n-i)!}=\sum_i 2^{2i+1}{m+1\choose 2i+1,n-i,m-n-i}
		$$
		
		\onslide<2->考虑组合意义,即在$m+1$个点中,给若干点标黑,若干点标白,可以不标,要求黑点比白点多$m-2n$个,每不标一个贡献乘$2$,求贡献和。
		
		\onslide<3->记$f_{x,y}$表示考虑了前$x$个点,黑点比白点多$y-x$个,则:
		$$
		f_{x,y}=f_{x-1,y}+2f_{x-1,y-1}+f_{x-1,y-2}
		$$
		
		\onslide<4->典型的格路记数DP方程(可以理解为先后走两步),$f_{x,y}$等于${2x\choose y}$。
		
		\onslide<5->所以答案为$f_{m+1,m-2n+m+1}={2m+2\choose 2m-2n+1}={2m+2\choose 2n+1}$		
	\end{frame}
	\clearpage
	\begin{frame}
		\frametitle{extra}
		\onslide<1->不难发现,$a_i=3i,b_i=3i+1$满足题意。
	\end{frame}
	\clearpage
\end{document}