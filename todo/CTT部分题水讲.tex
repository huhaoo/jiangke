\documentclass[10pt]{beamer}
\usepackage{xeCJK}
\usepackage{graphicx}
\usepackage{booktabs}
\usepackage{listings}
\usepackage{multirow}
\usepackage{mathtools}
\usepackage{ulem}
\usefonttheme[onlymath]{serif}
\usetheme{metropolis}
\begin{document}
	\title{CTT2021部分题水讲}
	\date{\today}
	\author{huhao}
	\maketitle
	\clearpage
	\begin{frame}
		\frametitle{D2T1}

		支持$3$个操作:全体对某一个数取$\min$,全体加下标,求一段区间和。

	\end{frame}
	\clearpage
	\begin{frame}
		\frametitle{D2T1}
	
		求每一个点第一次被取min时间即可。
	
	\end{frame}
	\clearpage
	\begin{frame}
		\frametitle{D2T2}
	
		令一个长度为$n$的01序列的$f$值为:$f(A)=\sum_{i=0}^{n-1}A_ip_i2^i$。

		保证$p_{n-1}=1,p_{n-2}=-1$,其它$p$都是$-1$或$1$。

		你可以询问两个序列$A,B$,会返回一个序列$C$,满足$f(A)+f(B)\equiv f(C) \pmod {2^n}$。

		$\log^2n$次询问求出$p$。
	
	\end{frame}
	\clearpage
	\begin{frame}
		\frametitle{D2T2}
	
		$f(1111111\dots)+f(100100100\dots)=?$
	
	\end{frame}
	\clearpage
	\begin{frame}
		\frametitle{D2T3}
	
		一个$n$个点的有向图,初始$i$向$i+1$连边,加$m$条边,可以有重边自环,求$1$到$n$随机游走期望步数最大值。
	
	\end{frame}
	\clearpage
	\begin{frame}
		\frametitle{D2T3}
	
		显然向$1$连边。

		求求导,发现后面的点连的边数$j$和前面连的边数$i$满足:$i\le j\le i+2$。

		再求求导,发现恰有一个比最后一个连的边少$2$。

		等比数列求和即可。
	
	\end{frame}
	\clearpage
	\begin{frame}
		\frametitle{D3T1}
	
		一棵$1$为根的树,给定一个排列$p:[2,n]\rightarrow [2,n]$,每一个点都是白的。

		按照排列顺序依次给$[2,n]$染黑。

		如果某一个时刻对于任意一个黑色的节点,子树内所有点都是黑的,那么就称这个时间是好的。

		求所有好的时刻,黑色连通块个数和。

		动态修改(加删边)这棵树。
	\end{frame}
	\clearpage
	\begin{frame}
		\frametitle{D3T1}
	
		$1$所在连通块数?
	
	\end{frame}
	\clearpage
	\begin{frame}
		\frametitle{D3T2}
	
		给定一个序列$a_{1\dots n},a_i\in[-1000,1001]$,且随机生成。

		询问若干$[l,r]$,求$[l',r']\subseteq [l,r]$下式最大值:

		$$
		\dfrac{1}{r'-l'+1}(\sum_{i=l'}^{r'} a_i)^2
		$$
	
	\end{frame}
	\clearpage
	\begin{frame}
		\frametitle{D3T2}
	
		对于一个$l$,$[l,r_1],[l,r_2],\dots$且依次递增的不会很多。
	
	\end{frame}
	\clearpage
	\begin{frame}
		\frametitle{D4T1}
	
		给定$n,p$,求最小的$k$,使得$n^{k+1}+1\bmod p=0$
	
	\end{frame}
	\clearpage
	\begin{frame}
		\frametitle{D4T1}
	
		求阶即可。
	
	\end{frame}
	\clearpage
	\begin{frame}
		\frametitle{D4T2}
	
		一个$n$个点的树,每个点上有若干个棋子,每次可以将一个点上面的一个棋子移到它的子树内。

		动态加棋子,每次询问一个点和和它相邻的点,有几个满足即使在任意一个点上放一个棋子,依然满足先手必胜。
	
	\end{frame}
	\clearpage
	\begin{frame}
		\frametitle{D4T2}
	
		考虑换根贡献即可。
	
	\end{frame}
\end{document}