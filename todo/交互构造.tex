\documentclass[10pt]{beamer}
\usepackage{xeCJK}
\usepackage{graphicx}
\usepackage{booktabs}
\usepackage{listings}
\usepackage{multirow}
\usepackage{mathtools}
\usepackage{ulem}
\usefonttheme[onlymath]{serif}
\usetheme{metropolis}
\begin{document}
	\title{构造 \& 交互}
	\date{\today}
	\author{huhao}
	\maketitle
	\clearpage
	\begin{frame}
		\frametitle{前天的题}
	
		三堆石头,每次可以将一堆石头$a$中部分石头移到另一堆$b$,移的数量必须与$b$中石头数量相等。

		将一堆石头移空。
	
	\end{frame}
	\clearpage
	\begin{frame}
		\frametitle{four stone}
	
		四个数$x,y,z,w$,每次可以选择两个数$a,b$,将它们变为$2b-a,b$。

		将它们变为$X,Y,Z,W$,顺序不论。
	
	\end{frame}
	\clearpage
	\begin{frame}
		\frametitle{NOIP2020}
	
		$n+1$个栈,栈的大小为$m$,前$n$个栈是满的。可以发现,一共有$nm$个数,其中分别有$n$个$1,\dots,m$。

		可以取出一个栈的栈顶,插入另一个栈。

		你需要通过若干次操作,使第$n+1$个栈依然为空,前$n$个栈第$i$个栈有$m$个$i$。
	
	\end{frame}
	\clearpage
	\begin{frame}
		\frametitle{操作的逆}
	
		后面两题显然可逆,第一题\textbf{似乎}不满足。

		操作可逆的话,就有如果$S$状态是可行的,$S\stackrel*\Rightarrow T$,那么有$T\stackrel*\Rightarrow S$,所以$T$也是可行的。

		\onslide<2->

		思考难度要求会变小:
		
		\begin{itemize}
			\item 不断简化,依然是可行,后一步可以用到前一步结果所保证的性质。
			\item 可以尝试用若干操作的复合:$f=\mathrm{op}_1 \circ \mathrm{op}_2 \circ \dots \circ  \mathrm{op}_k$。
		\end{itemize}
	\end{frame}
	\clearpage
	\begin{frame}
		\frametitle{有条件的逆操作}
	
		第一题先只考虑两堆石头,如果是一奇一偶(或者同时除以$2^k$),那么操作后依然是一奇一偶。

		一个状态$T$,经过操作变为$f(T)$的话,显然不存在$S\not=T,f(S)=f(T)$,因为每一个$T$都是可以找到唯一的逆(但不一定存在操作):将偶数除以$2$。

		所以$f$是一个满射,定义域和值域相等,所以$f$是双射。

		这样显然会成为若干个环,在环上绕一圈会回到原点,少走一步会走到$f^{-1}(T)$,在这种情况下存在逆!
	
	\end{frame}
	\clearpage
	\begin{frame}
		\frametitle{复合操作}
	
		第一题可以将$T\stackrel*\Rightarrow f^{-1}(T)$,作为一个复合操作。

		第二题可以将对于三个数$a,b,c$,把一个数$a$,平移$2|b-c|$作为一个复合操作。

		第三题可以将保证操作后第$n+1$个栈依然为空的前提下交换两个元素作为一个复合操作。

		这些操作保持了可逆性,所以任意操作都不会让本来可行的操作变得不可行。
	
	\end{frame}
	\clearpage
	\begin{frame}
		\frametitle{第一题}
	
		一次操作可能会将两个数大小不可控地改变,但是在复合操作中,无论初始如何,最后都会使原来的偶数变为一半。

		条件加强的话:如果原来的偶数是$4$的倍数,复合操作后它会变为$2$的倍数,依然是一个偶数。

		如果有两个偶数$a,b$和一个奇数$c$,如果$a\ge b$,那么先操作$f(<a,b>)=<a-b,2b>$,然后操作$f^{-1}(<2b,c>)=<b,b+c>$,这样会保持$a,b,c$的奇偶性,同时让$c$变大。

		不断让$c$变大直到$a,b$一个为$0$即可。

		其它情况可以通过若干次操作变为两偶一奇的情况。
	
	\end{frame}
	\clearpage
	\begin{frame}
		\frametitle{后两题}
	
		留作练习。
	
	\end{frame}
	\clearpage
	\begin{frame}
		\frametitle{优化思路}
	
		考虑更加特化的复合操作:步数少/功能强,但是对状态的要求变高。
	
	\end{frame}
	\clearpage
	\begin{frame}
		\frametitle{2021集训队互测 部分分}
	
		有$6k+3$个点的完全图,将这$\dfrac{(6k+3)(6k+2)}{2}$条边划分为$\dfrac{(6k+3)(6k+2)}{6}$个三角形。
	
	\end{frame}
	\clearpage
	\begin{frame}
		\frametitle{CTT2020 部分分}
	
		有$n$个点,连接$k$条边,使得不存在生成子图为左右点数均为$m$的完全二分图。

		令$p$为质数,解决:

		\begin{itemize}
			\item $n=p^2,k=p^3,m=2$。
			\item $n=p^3,k=p^5,m=3$。
		\end{itemize}
	
	\end{frame}
	\clearpage
	\begin{frame}
		\frametitle{2021集训队互测}
	
		给定$n,m$,满足$2m\ge n$,令$f(n,k)$为$0\sim 2^{n-1}$中二进制数为$1$个数为$k$的数的集合。

		求单射$g:f(n,m+1)\rightarrow f(n,m)$,满足$x~\operatorname{and} g(x)=g(x)$。
	
	\end{frame}
	\clearpage
	\begin{frame}
		\frametitle{怎么构造}
	
		条件看似不多,但构造间两两相互制约,一不小心就会出错。
	
	\end{frame}
	\clearpage
	\begin{frame}
		\frametitle{优秀的构造}
	
		取模系是一个十分优秀的构造:
		
		\begin{itemize}
			\item 模$k$意义下$x+0\sim x+k-1$互不相同。
			\item 模$p$意义下$0\sim x(p-1)$互不相同($x,p$互质)。
		\end{itemize}
	
	\end{frame}
	\clearpage
	\begin{frame}
		\frametitle{往取模上靠}

		构造函数$g$,然后让所有$g(S)\equiv 0 \pmod p$的$S$成为你的构造。

		证明可以固定若干变量然后解另一变量(下面会用例子解释)。
		
	\end{frame}
	\clearpage
	\begin{frame}
		\frametitle{第一题}
	
		将$6k+3$个点划分为$3$个大小为$2k+1$的集合,给所有点$0\sim 2k$标号。
		
		然后令$f(x,y,z)=x+y-2z$,对于所有$f\equiv 0$的$x,y,z$,让第$i$个集合内的编号为$x,y$,和第$i+1$个集合内编号为$z$的三个点组成三角形。

		可以发现,所有边都只会出现一次:不同集合内的$x,z$,都可以解出唯一的$y$,同一集合内的$x,y$,也会解出唯一的$z$。
		
		但是$x,y,z$不能相等,因为只有$f(k,k,k)=0$,所以所有不合法的三角形,所没连上的边,只有不同集合间的同标号的元素,用一个三角形给这些边连上即可。
	
	\end{frame}
	\clearpage
	\begin{frame}
		\frametitle{深入思考}
	
		如果$f'(x,y,z)=x+y-z$,这也是最直观的想法,不妨想想怎么由$f'$想到$f$。

		这样会出现边$(i,2i)$的遗漏,但是$(1,2),(2,4)$却没有$(4,1)$边会使得最后一步进行不了。

		但是如果缺的是$(i,i)$,就形成了若干个环。
	
	\end{frame}
	\clearpage
	\begin{frame}
		\frametitle{更进一步}
	
		如果有$12k+7$个点的话:

		给第一个点依次与剩下的点形成$6k+3$个三角形,这样可以得到$6k+3$个两两之间已经有边的二元组。

		先求出$6k+3$的一个解,如果$(x,y,z)$在解内,考虑给第$x,y,z$这三个二元组两两间连上边。

		$h(u,v,w)=u+v+w$即可,解$u/v/w$的过程就是证明。
	
	\end{frame}
	\clearpage
	\begin{frame}
		\frametitle{第二题}
	
		只说第一个subtask,第二个留作练习。

		将每个点与一个$n\times n$的矩阵上的格点一一对应,然后:

		$g(x,y)=x\cdot y$,$\cdot$是点乘。
	
	\end{frame}
	\clearpage
	\begin{frame}
		\frametitle{第三题}
	
		\sout{鸽,}留作练习。
	
	\end{frame}
	\clearpage
	\begin{frame}
		\frametitle{小总结}
	
		猜个生成方式写了就交,反正IOI赛制?
	
	\end{frame}
	\clearpage
	\begin{frame}
		\frametitle{交互题}
	
		啥都可以出交互题,所以简单讲讲一些常用方法吧。
	
	\end{frame}
	\clearpage
	\begin{frame}
		\frametitle{信息论}
	
		试证明仅支持查询偏序关系的情况下排序是$O(n\log n)$的。
	
	\end{frame}
	\clearpage
	\begin{frame}
		\frametitle{证明}
	
		$k$次询问只能区分出$2^k$个排列,所以最优是$O(\log n!)=O(n\log n)$的。
	
	\end{frame}
	\clearpage
	\begin{frame}
		\frametitle{前两天的题}
	
		$n$个人,每一个人可以看到前面每个人帽子的颜色,从后往前依次询问每一个人他自己帽子的颜色,回答可以被前面的人听见,最大化回答正确人数。
	
	\end{frame}
	\clearpage
	\begin{frame}
		\frametitle{solution}
	
		根据信息论,答案是$\dfrac 12n$!

		\onslide<2->

		但总的信息量是$2n-2$bit,只要让每一个人尽可能多的用上知道的信息即可。

		但是如果让后面的人点对点地告诉前面的人信息,就不可能做到$n-1$,可以通过简单的方式知道思路的可行性。
	
	\end{frame}
	\clearpage
	\begin{frame}
		\frametitle{CTT2020 试机}
	
		$n+1$个人,每个人有一顶帽子,上面写了长为$n$的01串,每个人只能看到别人的。

		每一轮,每一个人可以在一张大家都能读写的纸上写1的01字符,或是说出自己头顶上的帽子的01串。

		求最少轮数。
	
	\end{frame}
	\clearpage
	\begin{frame}
		\frametitle{solution}
	
		留作练习。
		
		提示:答案是2轮,考虑每一个人得到的信息。
	
	\end{frame}
	\clearpage
	\begin{frame}
		\frametitle{我的互测题}
	
		有根树,每次可以询问若干子树的并中点两两距离和,求出这棵树。
	
	\end{frame}
	\clearpage
	\begin{frame}
		\frametitle{另一道互测题}
	
		\sout{哪道是最水的互测题?}

		一个排列$p:[0,n-1]\rightarrow[0,n-1]$,可以操作使得$S\leftarrow S+2^{p_i}$,返回$S$的值为$1$的位数。
	
	\end{frame}
	\clearpage
	\begin{frame}
		\frametitle{solution}
	
		放出来给大家乐乐的,\sout{留作练习。}
	
	\end{frame}
	\clearpage
	\begin{frame}
		\frametitle{CTT2021 D2T2}
	
		令一个长度为$n$的01序列的$f$值为:$f(A)=\sum_{i=0}^{n-1}A_ip_i2^i$。

		保证$p_{n-1}=1,p_{n-2}=-1$,其它$p$都是$-1$或$1$。

		你可以询问两个序列$A,B$,会返回一个序列$C$,满足$f(A)+f(B)\equiv f(C) \pmod {2^n}$。

		$6$次询问求出$p$。
	
	\end{frame}
	\clearpage
	\begin{frame}
		\frametitle{solution}
	
		$f(11111111\dots)+f(100100100\dots)=?$。
	
	\end{frame}
	\clearpage
	\begin{frame}
		\frametitle{互测题}
	
		你有$\infty b$的内存,会调用$2^{26}$次,每一次你可以访问修改不超过$6$个位置,返回是否是最后一次。
	
	\end{frame}
	\clearpage
	\begin{frame}
		\frametitle{小总结}

		交互构造\sout{不会就是不会}没有什么普遍有效的方法,这里介绍了几个常用的思考路径,但是记住一个原则:简化问题!

	\end{frame}
\end{document}