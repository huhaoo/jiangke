\documentclass[10pt]{beamer}
\usepackage{xeCJK}
\usepackage{graphicx}
\usepackage{booktabs}
\usepackage{listings}
\usepackage{multirow}
\usepackage{mathtools}
\usepackage{ulem}
\usefonttheme[onlymath]{serif}
\usetheme{metropolis}
\setbeamercolor{footnote mark}{fg=blue}% set footnote color in beamer
\setbeamerfont{footnote}{size=\tiny}
\begin{document}
	\title{数据结构}
	\date{\today}
	\author{huhao}
	\maketitle
	\clearpage
	\begin{frame}
		\frametitle{前言}
	
		基本算法看情况跳过,除非很多人都不会。
	
	\end{frame}
	\clearpage
	\begin{frame}
		\frametitle{RMQ}
	
		复杂度很优秀,可以在卡时限的情况下考虑。
	
	\end{frame}
	\clearpage
	\begin{frame}
		\frametitle{CF1523H\footnote{https://codeforces.com/problemset/problem/1523/H}}
	
	\end{frame}
	\clearpage
	\begin{frame}
		\frametitle{分块}

		将序列分成大小为$W$的$\dfrac nW$块,使得每次询问和修改都是$O(\sqrt n\operatorname{poly} \log n)$。

		或是分为大小为$\dfrac{\log n}{2}$的若干块,预处理所有可能的不同块。\sout{应该不会考}
	
	\end{frame}
	\clearpage
	\begin{frame}
		\frametitle{Yuezheng Ling and Dynamic Tree\footnote{https://codeforces.com/problemset/problem/1491/H}}
	
		一棵$n$个点的有根树,每个点$i$的父亲为$f_i$,$f_i<i$,你需要支持:

		\begin{enumerate}
			\item 给定$l,r,x$,令$l\sim r$内所有$f_i$变为$\max(1,f_i-x)$。
			\item 给定$u,v$,求$\operatorname{lca}(u,v)$。
		\end{enumerate}

		复杂度要求$O(n\sqrt n)$。
	
	\end{frame}
	\clearpage
	\begin{frame}
		\frametitle{线段树}
	
		相信lazytag之类的大家都会。

		相信区间取$\min,\max$大家都会。

		相信插入一次函数求单点$\min$大家都会。

		相信维护每一段区间有多少前缀$\max$大家都会。

		\sout{讲完了。}
	
	\end{frame}
	\clearpage
	\begin{frame}
		\frametitle{CTT 2021D2T1}
		
		支持$3$个操作:全体对某一个数取$\min$,全体加下标,求一段区间和。
	
	\end{frame}
	\clearpage
	\begin{frame}
		\frametitle{树链剖分}
	
		修改查询相信大家都会。

		因为树链剖分的目的就是将树上问题转化为序列上问题,所以,有的树上dp题目,也可以使用树链剖分。
	
	\end{frame}
	\clearpage
	\begin{frame}
		\frametitle{互测题\footnote{时间不详,好像叫做黑白沙漠来着}}
	
		求一棵树,每个点有一个点权,对于每一个$k$,求点数为$k$的最大权独立集。
	
	\end{frame}
	\clearpage
	\begin{frame}
		\frametitle{点分治}
	
		\sout{相信大家都会}。

	\end{frame}
	\clearpage
	\begin{frame}
		\frametitle{LCT}
	
		相信大家也会。

		相信大家还会维护子树信息的。

		例题可以看CTT2021D3T1。

	\end{frame}
	\clearpage
	\begin{frame}
		\frametitle{路径问题}
	
		在树上,路径问题可以被看作是一个三元问题(要关心lca),主要有三种简化方案:

		\begin{enumerate}
			\item 点分治,变成若干集合的问题。
			\item 启发式合并。
			\item \sout{树链剖分,变为链上问题}。
		\end{enumerate}

		其中主要是第一种思路,如果复杂度比较紧,也可以考虑只要单$\log$的线段树合并。
	
	\end{frame}
	\clearpage
	\begin{frame}
		\frametitle{例}
	
		求树上有多少长度为$l$的路径。
	
	\end{frame}
	\clearpage
	\begin{frame}
		\frametitle{连通块问题}
	
		连通块个数为点减边的数量。
	
	\end{frame}
	\clearpage
	\begin{frame}
		\frametitle{CTT2021D3T1}
	
		一棵$1$为根的树,给定一个排列$p:[2,n]\rightarrow [2,n]$,每一个点都是白的。

		按照排列顺序依次给$[2,n]$染黑。

		如果某一个时刻对于任意一个黑色的节点,子树内所有点都是黑的,那么就称这个时间是好的。

		求所有好的时刻,黑色连通块个数和。

		动态修改(加删边)这棵树。
	
	\end{frame}
	\clearpage
	\begin{frame}
		\frametitle{莫队}
	
		基本的相信大家都会,进阶的:

		\begin{itemize}
			\item 树上莫队,\sout{想怎么做就怎么做。}
			\item 多维莫队,和二维一样,块大小$n^{\frac{d-1}{d}}$。
			\item 在线询问莫队,块大小$n^{\frac 23}$,存下$n^{\frac 23}$组信息,询问就用最近的那一个按普通莫队的方式询问。
			\item 不删除莫队,先移动右端点,再移动左端点,然后还原左端点。
			\item 二次离线,单次移动复杂度高,但是把$n^{1.5}$个移动的贡献一起算会简单很多。
		\end{itemize}
	
	\end{frame}
\end{document}