\documentclass[10pt]{beamer}
\usepackage{xeCJK}
\usepackage{graphicx}
\usepackage{booktabs}
\usepackage{listings}
\usepackage{multirow}
\usepackage{mathtools}
\usepackage{ulem}
\usefonttheme[onlymath]{serif}
\usetheme{metropolis}
\setbeamercolor{footnote mark}{fg=blue}% set footnote color in beamer
\setbeamerfont{footnote}{size=\tiny}
\begin{document}
	\title{杂题选讲}
	\date{\today}
	\author{huhao}
	\maketitle
	\clearpage
	\begin{frame}
		\frametitle{AGC056B\footnote{\url{https://atcoder.jp/contests/agc056/tasks/agc056_b}}}
	
		给定$n,m$和$m$给个区间,令长为$m$的序列$x$合法,当且仅当存在长为$n$的排列,且对于所有$i$,$p_{x_i}=\max\{p_{l_i},p_{l_i+1},\dots,p_{r_i-1},p_{r_i}\}$。

		求有多少合法的排列。

		$n\le 300$。
	
	\end{frame}
	\clearpage
	\begin{frame}
		\frametitle{solution}
	
		可以考虑$n$在哪个位置,然后就把这个序列划分为了两半,dp下去。

		但是会发现:$[1,2],[2,3],[3,4]$当$4$在$1$,$3$在$4$时,$x$序列和$3$在$1$,$4$在$4$情况相同。

		具体的,有交换$n$和另一个比较大的数答案不变的情况出现。

		不妨强制交换,使得$n$的位置尽可能靠右。

		不难发现,此时在$n$的右侧的下一个$n'$必须和$n$被同一个区间包含,dp即可,复杂度$O(n^4)$。
	
	\end{frame}
	\clearpage
	\begin{frame}
		\frametitle{AGC056C\footnote{\url{https://atcoder.jp/contests/agc056/tasks/agc056_c}}}

		给定若干区间,构造出一个01序列,满足所有区间中01个数相同。

		求字典序最小的方案。

	\end{frame}
	\clearpage
	\begin{frame}
		\frametitle{CF1299E\footnote{https://codeforces.com/problemset/problem/1299/E}}
	
	\end{frame}
	\clearpage
	\begin{frame}
		\frametitle{CF1548E\footnote{https://codeforces.com/problemset/problem/1548/E}}
	
	\end{frame}
	\clearpage
	\begin{frame}
		\frametitle{AGC032E\footnote{https://atcoder.jp/contests/agc032/tasks/agc032\_e}}
	
	\end{frame}
	\clearpage
	\begin{frame}
		\frametitle{solution}
	
		显然是划分为两半,前面一半对称连边,后面一半也是。

		显然划分点越前越好,找到最前的划分点即可。
	
	\end{frame}
	\clearpage
	\begin{frame}
		\frametitle{THE LIGHTS OF THE ROUND TABLE}
	
		$n$个灯泡分布在一个环上,一开始有部分亮并有部分暗,你每次可以指定某些位置灯泡,并改变它们的通电情况。

		不巧的是,在改变前,这个环都会以某种方式旋转若干角度,你必须依然改变那些位置上的灯泡。

		求是否有方案使得所有灯泡变亮。
	
	\end{frame}
	\clearpage
	\begin{frame}
		\frametitle{简单组合恒等式}
	
		$$
		\sum_i {m\choose 2i}{m-2i\choose n-i}2^{2i}
		$$
	
	\end{frame}
	\clearpage
	\begin{frame}
		\frametitle{简单证明题}
	
		$f_{i,j}=\max\{f_{i-1,j-1},f_{i-1,j},f_{i-1,j+1}\}+a_{i,j},f_{1,x}=[x=1]a_{1,x},f_{0,x}=f_{x,0}=f_{n+1,x}=f_{x,n+1}=\infty,\sum_{j}a_{m,j}=1,F=\min_i\{f_{n,i}\}$,求$F$的最大值(并证明)。
	
	\end{frame}
	\clearpage
	\begin{frame}
		\frametitle{AGC044C\footnote{https://atcoder.jp/contests/agc044/tasks/agc044\_c}}
	
	\end{frame}
	\clearpage
	\begin{frame}
		\frametitle{AGC030D\footnote{https://atcoder.jp/contests/agc030/tasks/agc030\_d}}
	
	\end{frame}
	\clearpage
	\begin{frame}
		\frametitle{AGC027D\footnote{https://atcoder.jp/contests/agc027/tasks/agc027\_d}}
	
	\end{frame}
	\clearpage
	\begin{frame}
		\frametitle{AGC023E\footnote{https://atcoder.jp/contests/agc023/tasks/agc023\_e}}
	
	\end{frame}
	\clearpage
	\begin{frame}
		\frametitle{AGC045B\footnote{https://atcoder.jp/contests/agc045/tasks/agc045\_b}}
	
	\end{frame}
	\clearpage
	\begin{frame}
		\frametitle{Project Euler 439\footnote{https://projecteuler.net/problem=439}}
	
		$$
		\sum_{<i,j>\in[1,n]^2}\sigma_1(i,j)
		$$
	
	\end{frame}
	\clearpage
	\begin{frame}
		\frametitle{CF1292E\footnote{https://codeforces.com/problemset/problem/1292/E}}
	
	\end{frame}
	\clearpage
	\begin{frame}
		\frametitle{CF848E\footnote{https://codeforces.com/contest/848/problem/E}}
	
	\end{frame}
	\clearpage
	\begin{frame}
		\frametitle{ICPC2022 Jinan G, Quick Sort}

		给定序列,求快排中swap次数。

		$n\le 5\times 10^5$
	
	\end{frame}
\end{document}